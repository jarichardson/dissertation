\documentclass[12pt,letter]{article}

%compile with pdflatex:
%:! bibtex %:r
%:! pdflatex -synctex=1 -interaction=nonstopmode --shell-escape %


\usepackage{amsmath}
\usepackage{natbib}
\usepackage{graphicx}

\title{Role of sills in the development of volcanic fields: Insights from lidar mapping surveys of the San Rafael Swell, Utah}
\date{}
\author{}

\usepackage[margin=1in]{geometry}
\usepackage{setspace}
%\doublespacing

\usepackage{lineno}
%\linenumbers

\begin{document}

\maketitle

\section{Abstract}

Analysis of airborne and terrestrial LiDAR data demonstrates that $>$0.4 km$^3$ of magma cooled in sills at shallow ($<$1 km) depth in the now eroded Pliocene San Rafael Swell distributed volcanic field, Utah. The volumes of each of seven sills are estimated from 3D models of the LiDAR data and range from 10$^{-4}$-10$^{-1}$ km$^3$. Directions of magma flow during emplacement are interpreted from precise sill thickness measurements and measurements of linear vertical offsets within the sills, helping to identify feeder conduits and dikes; 3D map relationships derived from LiDAR data demonstrate that magma flowed into and out of sills from these active dikes and eruptive conduits. Mapped sill volumes account for $>$92\% of intrusive material within the 50 km$^2$ study area. We conclude that sills played a significant role in modifying eruption dynamics during activity in San Rafael, and suggest that monitoring of sill inflation and deflation in active distributed volcanic fields may provide key information about unrest and potential eruption dynamics.


\section{Introduction}

Intermediate to shallow crustal storage of pre- and syn-eruption magma modulates magma supply rate in many volcanic systems. At Mount Saint Helens (1980, USA) and Parícutin (1943, Mexico), magma supply rate is thought to have been influenced by the presence of shallow (<10 km), temporary magma storage (Cashman and McConnell, 2005) and by the length of storage time (Scandone et al., 2007). Erlund et al. (2010) identified increasing amounts of shallow crust (≤4 km depth) assimilated at Parícutin over the 9-year eruption, and concluded that a shallow intrusion network formed early and caused later eruptive products to be more effusive. On shorter timescales, gas may segregate preferentially into conduits above shallow sills, increasing volumetric flow in the conduit and intensifying the eruption (Conte, 2000; Pioli et al., 2009). Sill-like intrusions into shallow magma chambers have recently been geodetically linked with InSAR and seismic monitoring to eruptions at Tungurahua, Ecuador (Biggs et al., 2010) and Eyjafjallajökull, Iceland (Tarasewicz et al., 2012). These models and observations suggest that it is critical to understand the volume, depth and distribution of sills in volcanic fields in order to forecast eruption dynamics and the evolution of volcanic systems. In young volcanic fields, such as the one around Parícutin (Connor, 1990), it is not possible to directly observe the shallow plumbing system. Here, we use LiDAR (Light Detection and Ranging) technology to map part of the eroded San Rafael Swell (UT) volcanic field. These data demonstrate that sills are prevalent at shallow depths (< 1 km), modulated magma flow in eruptive conduits, and likely influenced eruption dynamics within this volcanic field.


\section{Geologic Description}

The San Rafael volcanic field was active between 4.6 and 3.8 Ma (Delaney and Gartner, 1997). This volcanic field is part of a larger occurrence of Cenozoic basaltic volcanism in the Colorado Plateau and Basin and Range provinces but is distinct from many other fields as it has been eroded to a depth of $\sim$800 m, based on its age and late Cenozoic erosion rates (e.g., Pederson et al., 2002). The sill and dike swarm, or volcanic plumbing system, cut a Jurassic sedimentary section from the Carmel Formation through the Cutler Formation. Diabasic dikes in this area trend 335° to 0° N along regional joint sets, indicating low horizontal deviatoric stress during emplacement (Delaney and Gartner, 1997). Sills in the San Rafael Swell range from <5 m to >40 m thick and are exposed in cliff sides and canyons in outcrops that extend for 100s-1000s m. This shallow magma plumbing system has been mapped (Delaney and Gartner, 1997), and used to improve our understanding of (i) dike emplacement (Delaney et al., 1986), (ii) magma diapirism in the shallow crust (Diez et al., 2009), and (iii) the spatial relationships between dikes and conduits (Kiyosugi et al., 2012), which are often surrounded by brecciated country rock, indicating conduit erosion during rapid magma ascent. Although Gartner (1986) described physical characteristics of exposed sills in the area, the complex emplacement processes of sills in this volcanic field have remained enigmatic. With the aid of LiDAR, we are able to document the complex map relationships between intrusions in the area.

\section{Lidar Reconnaissance and Analysis}

Terrestrial Laser Scanning (TLS), performed in 2010 and 2012, collected a 7.3 GB point cloud over an area of 5 km$^2$. Both TLS surveys used Riegl terrestrial scanners. An airborne laser scanning (ALS) survey, in 2013, provided data for a 54 km$^2$ Aerial Laser Swath Map (ALSM) (Richardson, 2013), connecting the 2 TLS surveys into a single study area (Fig. 1). Instrument specifications and data formats from these surveys are outlined in Table S1. Co-registering the coordinate systems of the three surveys creates a tabular block $\sim$50 km$^2$ in area, within which relief varies by up to 500 m. Thus, we are able to characterize the magma plumbing system in a volume approaching 25 km$^3$, which bounds the current study area extent.

LiDAR provides high-density altimetry and infrared (IR) albedo data, which aids in distinguishing intrusions from the country rock. Because the 3D point cloud is so precise, LiDAR data help identify subtle changes in sill thickness over large areas, vertical offsets in sills (even where the offset itself is covered), and disrupted stratigraphy in overlying sedimentary units, which allow magma movements to be deduced. This use of LiDAR to reconstruct a magmatic plumbing system enables us to model the amount of magma emplaced into the crust due to Pliocene volcanism for the 25 km$^3$ space bounding the combined surveys.

The three point clouds (two TLS and one ALSM) were consolidated and analyzed using LiDAR Viewer (LV) (Kreylos et al., 2008). Contacts between igneous and sedimentary rocks were identified by shade contrast (igneous rocks are generally darker than sedimentary rocks in near-IR) and weathering patterns easily observable in the point cloud (Fig S2). Thickness measurements are made in LV where sill upper and lower contacts are seen in close proximity. The exact locations of sill contacts are manually picked between points in the point cloud, where one point is interpreted as sill and the other as sedimentary rock (Table S2).  Uncertainty at each measurement is determined as the average of point-to-point distances on top and bottom of the sill and is drastically reduced in areas where both TLS and ALSM data are available. Other measurements made in LV include sill base elevations, strikes and dips of continuous sill segments and of sedimentary host rock below sills. Locations where sills abruptly change stratigraphic level are also mapped in the field. These abrupt changes can be traced between outcrops with point cloud measurements made at inaccessible regions within the study area.

Sill exposures are mapped using 1 m National Agriculture Imagery Program (NAIP) images and the ALSM Digital Elevation Model. These are combined with thickness measurements to estimate terminal boundaries of sills. Sill volume and average thickness are modeled by constraining the thicknesses of sills at their respective modeled boundaries to be 0 m thick and interpolating a Laplacian-spline surface within sill boundaries, calibrated to the measured thicknesses. Results from this, mapped areas and average measured thicknesses are detailed in Table 1.

Approximately 53 km of dikes and 16 conduits in the study area have been mapped by Kiyosugi et al. (2012) and are included in our reconstruction of the magmatic plumbing system. The cumulative volume of igneous material stored in dikes is estimated to be the product of dike length, the modeled block height, and 85 cm, the modal dike thickness (Delaney and Gartner, 1997). This might be a slight overestimate as some dikes might not have cut through the entire block height. The volume stored in conduits is the product of the surface area of each conduit and the modeled block height. The surface areas of conduits are determined by mapping their outlines with the ALSM DEM and NAIP images. This assumes conduit thickness does not change within the vertical limits of this reconstruction and might underestimate volume if conduits formed above the present day surface or if conduits widen toward the surface.

\section{Igneous System Reconstruction}

Seven isolated sills crop out within the study area. We interpret these sills to have been emplaced independently as a result of single dike injections, based on evidence described below. Sill volumes range from 10$^{-4}$-10$^{-1}$ km$^3$ and have been emplaced over areas of 10$^{-1}$-10s km$^2$ (Table 1). Through modeling sill geometries, we find that $\sim$0.4 km$^3$ of igneous material is permanently stored in the sills, representing 93\% of all intrusive rocks in our reconstructed volume. Table 2 summarizes mapped areas and modeled volumes of sills, dikes, and conduits within the study area. By combining adjacent conduits along the same dike, we estimate that 12 distinct volcanic events are represented within the study area. Emplacement processes of sills and their role in the development of the Pliocene volcanic field can be further understood by investigating individual sills.

\subsection{Hebes Sill}

The sill at Hebes Mountain (Fig. 1) is primarily preserved as a single 1.9 km$^2$ sill exposed over an area of $\sim$4 km$^2$. This sill generally dips with strata 1–8° to the northwest, although locally some areas dip 5–30° toward the center of the sill. Sill thicknesses are measured to a precision of ± 75 cm, with virtually all exposures measuring >19 m. The sill thins monotonically from the center to the edges of Hebes Mountain, thinning most rapidly to the southwest.

By modeling Hebes sill as a 5.4 km$^2$ area, roughly following the shape of Hebes Mountain, the volume is estimated to be $8.5\cdot 10^{-2}$ km$^3$. The elongate nature of this sill model (Fig S1, top right) with increased thickness trending in the northwest dip direction is aligned in the regional dike direction, perhaps indicating a linear source region (dike) feeding the sill.

\subsection{Central Cedar Sill}

The Central Cedar Sill caps two buttes to the east of Cedar Mountain and is exposed on the east facing cliffs of Cedar Mountain (Fig. 3). The outcrops are interpreted to be parts of a single sill, as their basal contacts project across the small valleys between each exposure at the same elevation. Measured thicknesses of the Central Cedar Sill are 2–15 m, with basal contacts that dip with sedimentary host rock, at 2–5° WNW to SW. The sill outcrops adjacent to a conduit associated with a $\sim$2 km long dike on Central Cedar Mountain. Basalt between the conduit and sill appears continuous, with no brecciation, suggesting the dike and sill were formed coincident, and were thus comagmatic.

The average uncertainty in thickness measurements for Central Cedar Sill is <20 cm, due to coverage from both ALSM and TLS data sets. The point cloud also enables the mapping of curvilinear “step-up” features, defined by Gartner (1986) as vertical offsets between different intrusion pathways, or feathers. Flow direction during intrusion is interpreted to be parallel to step-ups. Step-ups in this sill indicate flow to the W–WNW, away from and/or toward, the conduit. Modeling this sill as a tongue-shaped body intruding to the west from the suspected source dike (Fig S1, top center), Central Cedar Sill has an aerial extent of $\sim$0.88 km$^2$, and a total volume of $4.4\cdot 10^{-3}$ km$^3$ (Table 1).

A linearly thinning trend away from the conduit is evident in the sill, continuing for 1 km to the observed sill limit (Fig. 4). Within 100 m of the conduit, sill thickness changes dramatically due to rotated sandstone blocks with thin basalt lenses injected over the tops of the sandstone blocks, indicating roof collapse into the sill (Fig. 3). From these observations we conclude that the Central Cedar Sill was fed from a single dike and was emplaced in a tongue-like fashion to the west in its initial dipping direction. Further, we infer that a conduit-forming volcanic event may have halted further advance of the sill and subsequent flow of magma from the sill into the conduit caused the observed conduit-adjacent roof collapse.


\section{Discussion and Conclusions}

Through LiDAR mapping of the San Rafael study area, 7 sill-forming events in the shallow crust and 12 conduit-forming events have been identified and mapped in detail (Fig. 1). We model the total volume of igneous material stored in sills to be 0.4 km$^3$ within a 25 km$^3$ block. This sill volume represents 93\% of the stored igneous volume in the block, with the remaining 7\% in dikes and conduits. There is no doubt that, volumetrically, sills are a critical component of the magma plumbing system in this distributed volcanic field.

It is possible, in fact, that sill volume in the San Rafael Sell volcanic field is comparable to erupted volume. Eruption volumes for the 12 conduits cannot be directly observed, as those lavas are completely eroded away. Eruption volumes for monogenetic volcanoes in similar fields span three orders of magnitude, ranging from 10$^{-3}$-1 km$^3$(e.g., Crowe et al., 1983, Condit et al., 1989, Kiyosugi et al., 2010). If we assume that average eruption volume is 0.1 km$^3$ for conduits in the San Rafael Swell, $\sim$1.2 km$^3$ of basalt would have been erupted at the surface, four times the estimated sill volume. Again, this comparison suggests that, volumetrically, crustal storage of magma in sills is a major feature of the magmatic system.

Sills at this depth (< 1 km) are intruded well above their level of neutral buoyancy. This is evident in both the Hebes and Central Cedar Sill (Fig. 4) where propagation away from an apparent source is down-dip. Sills in this area generally ascend stratigraphy only after lifting the roof, enabling exploitation of new bedding planes and initial emplacement at this depth is a pressure driven process, similar to the intrusion of the Trachyte Mesa laccolith of the Henry Mountains (Utah, USA) (Wetmore et al., 2009) and shallow sills at the Paiute Ridge (Nevada, USA) (Valentine and Krogh, 2006). To inject magma horizontally at this depth, the work necessary to break and lift the crust must be less than the work needed to break and separate above rocks, in addition to the work required to support the column of negatively buoyant magma stored in the source dike (Lister and Kerr, 1991). At this depth ($\sim$1 km), lithostatic pressure produced by the sedimentary rocks was $\sim$20 MPa and was likely a comparable work load to driving magma toward the surface as a dike. Because deviatoric stress was low in this area during the Pliocene (Delaney et al., 1986), the direction of least compressive stress could have significantly migrated from horizontal, thus enabling sill formation along with dike ascent (Rubin, 1995).

The development of shallow sills likely affected eruption dynamics. If a comagmatic sill is present during a volcanic eruption, ascending bubbles can become concentrated in the vertical conduit at the conduit-sill junction by disproportional capture of the liquid phase of a two-phase flow in the horizontal branch (Conte, 2000). This concentration occurs if overall magma flux is sufficiently low. The presence of the sill, therefore, enables modulation of explosive potential, with low magma flow rates resulting in more explosive activity than if a sill was not present. Following the method of Pioli et al. (2009), assuming a magma density of 2800 kg/m$^3$, we calculate the transition flux to be $1.8\cdot 10^5$ kg/s within the Central Cedar Mountain conduit (diameter 25 m), where lower flux will concentrate bubbles in the conduit system. As average mass eruption rate for strombolian eruptions is commonly observed to be 10$^3$-10$^5$ kg/s (Pioli et al., 2009), the presence of sills at the level where H$_2$O exolves critically impacts volcanic hazard.

Under some distributed volcanic fields, the primary storage of magma in the shallow crust might be periodically emplaced, ephemeral magmatic sills, such as those in the San Rafael Swell. Similar sill emplacement events in active fields might be detected through volcano monitoring technologies, such as InSAR (Biggs et al., 2010). The observation of comagmatic conduits, dikes, and sills suggests that in such shallow crustal exposures the ascent of magma was likely quite rapid. The variability in emplacement processes of magma in the shallow surface can inform advanced monitoring efforts in distributed volcanic fields. This study shows that periods of unrest will be more frequent than eruptions.



%\begin{figure}
%\centering
%\includegraphics[width=\linewidth]{map_diff}
%\label{fig:map_diff}
%\end{figure}


\end{document}
