\documentclass[12pt,letter]{article}

\usepackage{amsmath}
\usepackage{natbib}
\usepackage{graphicx}


\usepackage{booktabs}
\newcommand{\tabitem}{~~\llap{\textbullet}~~}

\setcounter{section}{-1}
\makeatletter
% we use \prefix@<level> only if it is defined
\renewcommand{\@seccntformat}[1]{%
  \ifcsname prefix@#1\endcsname
    \csname prefix@#1\endcsname
  \else
    \csname the#1\endcsname\quad
  \fi}
% define \prefix@section
\newcommand\prefix@section{Level \thesection: }
\makeatother

\title{A Benchmark Hierarchy for lava flow simulations}
\date{}
\author{}

\usepackage[margin=1.5in]{geometry}
\usepackage{setspace}
%\doublespacing

%\usepackage{lineno}
%\linenumbers

%Geology Papers are limited to ~5000 words

\begin{document}

\maketitle

\section{Conservation}
\subsection{Conservation of Mass}\label{test:CoM}
	Before the results of a lava flow simulation can be benchmarked or validated, it must be verified to at least prove that conservation of mass is preserved. A lava flow simulation will therefore not be tested against the following benchmark tests until this conservation of mass requirement is shown to be fulfilled.

\begin{center}
	\begin{tabular}{l}
		\toprule
		\textbf{Test \ref{test:CoM} Algorithm}\\
		\midrule
		Implemented in MOLASSES:\\
		1.~Count lava delivered to vents, $V_{in}$\\
		2.~At end of flow, count all lava in all active cells, $V_{out}$\\
		3.~Test $V_{in}-V_{out}=0$\\
		~\tabitem \textit{True}: Success\\
		~\tabitem \textit{False}: Failure
		\bottomrule
	\end{tabular}
\end{center}

%Level 1
\section{Self similarity given ersatz parameter space variation}
\subsection{Rotating Sloped DEM}\label{test:SS_rotate}
The DEM rotation scheme by Miyamoto and Sasaki (1997) is adopted and expanded, so that a DEM with a simple sloping face is rotated 180 times at an increment of 2$^{\circ}$. The farthest point of a simulated flow from a given model is reported for each slope direction. A perfect flow model will have no variation in distance traveled with respect to slope direction.

\begin{center}
	\begin{tabular}{l}
		\toprule
		\textbf{Test \ref{test:SS_rotate} Algorithm}\\
		\midrule
		1.~Assing constant DEM and Flow parameters\\
			~\tabitem $\phi:=$DEM slope\\
		2.~Write configuration file\\
		3.~For 180 azimuths, $\theta$, from 0-360$^{\circ}$\\
			~\tabitem create DEM dipping $\phi$, $\theta$ from N.\\
			~\tabitem run \textbf{MOLASSES} over DEM\\
			~\tabitem $l$:=i,max dist$\{C_0-C_i\}$\\
			~\tabitem $d_{l,\theta}$:=dist$\{C_0-C_l\}$\\
			~\tabitem $w$:=i,max dist$\{\overrightarrow{C_0C_l}-C_i\}$\\
			~\tabitem $d_{w,\theta}$:=dist$\{\overrightarrow{C_0C_l}-C_w\}$\\
		4.~Find average, $\mu$, standard deviation, $\sigma$ of $d_l$,$d_w$\\
			~\tabitem CV$_l$:= $\sigma_l/\mu_l$\\
			~\tabitem CV$_w$:= $\sigma_w/\mu_w$\\
		5.~Define success, failure based on CV.
		\bottomrule
	\end{tabular}
\end{center}

\subsection{Scaling Spatial Resolution}\label{test:SS_resolution}
We propose a second DEM-altering test, where lava flows will be simulated over two two sloped surfaces. The first surface will be divided into grid cells with edges half the length of the second surface cell edges.


%Level 2
\section{Replication of flow morphologies on simple physical surfaces}

\subsection{Flow areal extent approximates a circle on a flat plane}\label{test:Bing_circ}
Here we measure flow algorithm performance on a flat surface from a single vent source location. To measure the extent to which the simulated flow replicates a circle, we measure the inundated area and compare that to the area of a circle which circumscribes the flow exactly. This can be described as
\begin{equation}
Fit = \frac{A_{flow}}{\pi d_{max}^2}
\end{equation}
where $d_{max}$ is the farthest extent of the simulated flow from the vent. A perfect match to a circle would result in a $Fit=1$. 

\begin{center}
	\begin{tabular}{l}
		\toprule
		\textbf{Test \ref{test:Bing_circ} Algorithm}\\
		\midrule
		1.~Create flat DEM with spatial resolution, $R$\\
		2.~Write configuration file for flow\\
		3.~Run \textbf{MOLASSES} over DEM\\
		4.~Define results:\\
			~\tabitem $d_{max}$~:=~max dist$\{\overrightarrow{C_0C_l}-C_i\}$\\
			~\tabitem $N$~:=~Number of cells, $C$, inundated\\
		5.~Compare $N$ with $N_{circle}$,$N_{octagon}$,$N_{square}$ where:\\
			~\tabitem $N_{circle}:=\pi d_{max}^2/R$\\
			~\tabitem $N_{octagon}:=2.82842 d_{max}^2/R$\\
			~\tabitem $N_{square}:=2d_{max}^2/R$\\
			~1. If $N>N_{octagon}$: Success\\
			~1. If $N<N_{square}$: Failure\\
		\bottomrule
	\end{tabular}
\end{center}

\subsection{Flow thickness profile approximates a Bingham fluid on a flat plane}\label{test:Bing_thick}
The second flat-surface test compares the averaged flow thickness profile away from the vent from a theoretical axisymmetric thickness profile of a viscoplastic fluid. The analytical solution for height with respect to radius is given by Griffiths (2000) as
\begin{equation}
h(r)^2 = \frac{2\sigma_0(R-r)}{\rho g}
\end{equation}
where $\sigma_0$ is the yield strength of the flow, R is the outer radius, and $\rho$ is the fluid density. 

\subsection{Flow thickness profile approximates a Bingham fluid on a slope}\label{test:Bing_slope}
The third test at Level 2 will be for recreating viscoplastic flow morphology for a flow on a slope, following Osmond and Griffiths (2001)


%LEVEL 3
\section{Replication of real lava flows over complex topography}
\subsection{2012-3 Tolbachik, Russia lava flows}\label{test:Real_Tolbachik}
The goodness of fit between a simulated flow and a mapped lava flow can be simply measured by the Jaccard coefficient. 

\begin{center}
	\begin{tabular}{l}
		\toprule
		\textbf{Test \ref{test:Real_Tolbachik} Algorithm}\\
		\midrule
		1.~$C_{real}$~:=~Locations of grid cells in Tolbachik DEM inundated by 2012-3 Flow.\\
		2.~Run \textbf{MOLASSES} over Tolbachik DEM\\
		3.~Define results:\\
			~\tabitem $C_{model}$~:=~Locations of cells, $C_i,...,C_N$, inundated by \textbf{MOLASSES} flow.\\
		4.~$Fit:=\cap_C/\cup_C$\\
			~\tabitem If $Fit\ge0.5$: Success\\
			~\tabitem If $Fit<0.5$: Failure\\
		\bottomrule
	\end{tabular}
\end{center}

\subsection{Glass Pours, Tampa?}\label{test:Real_Glass}
\subsection{Nornahraun, Iceland?}\label{test:Real_Iceland}

\end{document}
