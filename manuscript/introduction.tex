\chapter[Introduction]{Introduction}

Distributed volcanism is a style of volcanic activity that forms clusters of edifices instead of building one large edifice in successive eruptions through a central vent \citep{valentine2000basaltic}. Each edifice in a volcanic field, often a scoria cone, lava dome, or maar diatreme, is most often created from the arrival of a single magmatic dike to the surface during a single eruptive phase that lasts from weeks to decades. The term ``monogenetic volcano,'' used to describe scoria cones, small shield volcanoes, and maar volcanoes, comes from this concept of single eruptions forming small volcanoes \citep{greeley1977basaltic}, though individual scoria cones sometimes exhibit recurring volcanic activity over tens to hundreds of years \citep{hill1998cerro}. Distributed-style volcano clusters, forming from dikes ascending through lithosphere from a magma source hundreds to many thousands of square kilometers in extent, are reflections of properties of both their source region and the lithospheric filter \citep{settle1979structure}.

People have lived within volcanic fields since pre-historic times and their livelihoods have been affected when eruptions occur, usually with little to no warning \citep{elson2007living}. Recently, spacecraft have been sent to other planets, revealing the existence of distributed-style volcanism on Mars \citep{carr1977some} and Venus \citep{head1992venus}. On these neighboring planets, tens to tens of thousands of distributed volcanic vents form patchworks of shield volcanoes \citep{richardson2013volcanic,miller2012shield}. 

This dissertation assembles five projects that explore different aspects of distributed volcanic fields. The first project focuses on the role of the magma plumbing system on evolution of volcanic fields. I use lidar datasets to model magmatic sills in the San Rafael Swell, Utah \citep{richardson2015sills}. Second, a lava flow simulator is validated and used to model a recent lava flow. This simulator has modeled the 2012-3 Tolbachik lava flow in Kamchatka, Russia, using flow characteristics derived from TanDEM-X InSAR data \citep{kubanek2015lava}. Third, two chapters focus on volcano clusters on Mars, characterizing the long-term evolution of two fields. The first of these (Chapter \ref{ch_syria} studies the Syria Planum region \citep{richardson2013volcanic}. The second (Chapter \ref{ch_arsia}) focuses on Arsia Mons \citep{richardson2015recurrence}. The final chapter models and compares the spatial density of volcanic vents in clusters on Earth, Mars, and Venus \citep{richardson2012comparison}.

\section{Role of sills in the development of volcanic fields}
In Chapter \ref{ch_sills}, terrestrial and airborne lidar data are used to map volcanic features in the San Rafael Swell (Utah, USA). The San Rafael Volcanic Field is a Pliocene-aged volcanic field that has been eroded to a depth of $<$1~km exposing the igneous intrusion network. Analysis of the combined lidar datasets enables modeling of the geometries of seven magmatic sills. The total volume of these sills is $>$0.4~km$^3$, with each sill containing 10$^{-4}$-10$^{-1}$ km$^3$ of igneous rock. Mapped sill volumes account for $>$92\% of intrusive material within the 25~km$^3$ block that geometrically bounds the study area, with the rest of the material being stored in dikes and volcano conduits.
Sills in the San Rafael likely played a significant role in modifying eruption dynamics. At least one sill formed cotemporally with an eruption. Depending on the conduit diameter and the adjacent co-magmatic sill height, gas would have been entrained in either the conduit or the sill, modulating eruption explosivity.

\section{Validating lava flow simulators using a validation hierarchy and bayesian analysis}
In Chapter \ref{ch_molasses}, a cellular automata \citep{wolfram1984cellular} lava flow simulator is developed following the algorithm of \citet{connor2012probabilistic}. The simulator, named MOLASSES, has been developed in C with a modular framework, which enables users to quickly change relatively few lines of code to modify the flow behavior.

Validation exercises have been designed with the objective of evaluating simulated lava flows against the 2012-3 Tolbachik lava flow (Kamchatka, Russia). Two Bayesian posterior statistics, the Positive Predictive Value and the Negative Predictive Value are introduced to measure agreement between simulated and observed lava flows. These metrics are used to provide insight into improving model performance, and one model parameter in MOLASSES is optimized by finding the maximum of both predictive values. These two scores are also used to characterize the lava flow simulator when elevation uncertainty is taken into account. This provides a range of simulations which give a probabilistic model of lava inundation in the Tolbachik area.

\section{The volcanic history of Syria Planum, Mars}
In Chapter \ref{ch_syria}, a field of distributed shield volcanoes in the Syria Planum region of Mars is mapped to determine abundance, distribution, and alignments of vents. Nearest neighbor and two--point azimuth analyses are conducted, using mapped volcanic vent locations, to assess the spacing and orientations between vents across the study area. Two vent fields are identified as unique volcanic units along with the previously identified Syria Mons volcano. Superposition relationships and crater counts indicate that these three volcanic episodes span $\sim$900 Ma, beginning in the early Hesperian Period and ending in the Early Amazonian Period. No clear hiatus in eruptive activity is identified between these events, as crater age-dating error bars of each unit abut the modeled ages of other units. However, activity migrates from eruptions at Syria Mons, to regionally distributed eruptions that form the bulk of the Syria Planum plains, to dispersed eruptions in Syria's northwest. 

Nearest neighbor analyses suggest a non--random distribution among the entire population of volcanoes comprising Syria Planum, which is interpreted to result from the interaction of independent magma bodies ascending through the crust during different stress regimes throughout the region's eruptive history. Two--point azimuth results identify three orientations of enhanced alignments, which match well with radial extensions of three major tectonic centers to the south, east, and northwest of the study area. 

As such, Syria Planum volcanism evolved from a central vent volcano to dispersed shield field development over several hundred million years, an extremely long timefrime compared with Earth volcanism, during which the independent magma bodies related to each small volcano interacted to some extent with one or more of at least three buried tectonic patterns in the older crust.


\section{Waning volcanism on Arsia Mons, Mars}
In Chapter \ref{ch_arsia}, the recurrence rate of a volcanic field in the 110~km caldera of Arsia Mons, Mars, is modeled by combining stratigraphy and crater retention rate age modeling. In this caldera, 29 volcanic vents have been mapped and each have long lava flows extending several to tens of kilometers downhill. Vents in this caldera are comparatively young ($\sim$130~Ma on average), since no craters in the floor are larger than 1~km in diameter. The age of each vent can be modeled with crater counts, but the age uncertainty associated with this method can be larger than the total age of the field. To better quantify crater age model uncertainty, stratigraphic information has been added, since each lava flow embays or is embayed by at least one adjacent flow.

An algorithm has been created to identify potential ages of each vent based on crater age models and stratigraphy. This algorithm, named the Volcanic Event Recurrence Rate Model (VERRM), is then used in a Monte Carlo fashion to create 100,000 possible vent age sets. The recurrence rate of volcanism with respect to time is calculated for each possible age set, and these rates are combined to calculate the median recurrence rate of all simulations. This method finds that, for the 29 volcanic vents, distributed volcanism likely began within the caldera around 200~Ma then peaked around 150~Ma, with an average production rate of 0.25~vents per Ma. Volcanism then waned until the final vents were produced 10-90~Ma.

By applying modeled volumes to each volcano, volume flux rate is also calculated. Depending on the estimated volume, volume flux might have reached a peak rate of 0.1-3 km$^3$~Myr$^{-1}$ by 150~Ma and sustained this rate until about 90~Ma, when the volume flux dimished greatly.

The onset of volcanism of the 29 vents in the caldera at $\sim$200~Ma can be related to volcanic ash that was emplaced before 200~Ma on the flanks of Arsia Mons. The close timing of evidence of large, explosive volcanism and the oldest of the distributed effusive volcanoes might indicate that around 200~Ma the style of volcanism at Arsia Mons transitioned from explosive to effusive. If this transition took place, then the waning of recurrence rate of volcanism since 150~Ma might be the termination of larger magmatic activity related to construction of the Arsia Mons edifice.


\section{Uses of kernel statistics on volcanic vents}
Chapter \ref{ch_kde} compares the spatial intensity of volcanic vents (vents per unit area) for volcano clusters on the Earth, Mars, and Venus, revealing a fundamental difference between clusters on the three planets. On Earth, vents in volcano clusters are packed at 0.1 vents~km$^{-2}$. On Mars, vents in clusters are two orders of magnitude more dispersed, at 0.001 vents~km$^{-2}$. On Venus, clusters have an intermediate density of 0.01 vents~km$^{-2}$. This change in distribution scale is due to differences in the structure and composition of the planets' lithospheres, as well as the way broad regions of magma generation are created.

Spatial intensity is modeled with Kernel Density Estimation, a non-parametric statistical tool that models point density as a 2D density distribution by assigning probability density functions (PDFs) to a population of mappable points. Volcanic vent locations from several catalogs are used to compare cluster characteristics across the Solar System, including vent catalogs created and used in previous chapters (Arsia Mons Caldera vents in Chapter \ref{ch_arsia}, Syria Planum vents in Chapter \ref{ch_syria}, and San Rafael volcanic conduits used in Chapter \ref{ch_sills}). In total, 20 vent clusters are used: 10 volcanic fields on Earth, 3 fields on Mars, and 4 shield fields and 3 shield plains on Venus. 

The influences of the lithosphere and magma source region on dike formation and ascent are reflected in the distribution of volcanoes at the surface of a volcanic field. Here, Kernel Bandwidth (i.e. the width of the PDF used to model vent density for a volcano cluster) is used as a proxy of cluster elongation and orientation. Bandwidth characteristics are compared in regions where elongated source regions are expected, where the source region might have migrated over time, and where lithospheric features might enable or inhibit magma focusing in different directions.

%%%%%%%%%%%%%%%%%%%%%%%%%%%%%%%%
%References
%%%%%%%%%%%%%%%%%%%%%%%%%%%%%%%%

%%%%%%%%%%%%%%%%%%%%%
%%References Section
%\section{References}

%\nobibliography{dissertation_refs}
%\bibliographystyle{apalike} 
