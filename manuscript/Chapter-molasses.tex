\chapter[Validating Lava Flow Simulators using a Validation Hierarchy and Bayesian Analysis]{Validating Lava Flow Simulators using a Validation Hierarchy and Bayesian Analysis}\label{ch_molasses}

%compile with pdflatex:
%:! bibtex %:r
%:! pdflatex -synctex=1 -interaction=nonstopmode --shell-escape %

\renewcommand*{\FigPath}{figures/chapter-molasses}

\section*{Abstract}
	Modeling lava flows through cellular automata (CA) methods enables a computationally inexpensive means to quickly forecast lava flow paths and ultimate areal extents. A CA program has been created in the program language C that is modular, which enables a combination of governing CA rules to be evaluated against each other. My objective is to find a successful combination of automata behaviors that behaves like a bingham fluid and accurately forecasts lava inundation. To fulfill this objective, four validation levels have been devised, into which different tests can be applied to evaluate lava spreading algorithms against increasingly complex tests. These levels are 1) verification of the code by testing for conservation of mass; 2) testing for flow self-similarity given inconsequential variations in input parameters; 3) testing for replication of Bingham flow morphology on simple surfaces; and 4) testing for replication of real lava flow morphologies on pre-eruption elevation models. Two Bayesian posterior statistics, the Positive Predictive Value ($\text{Pr}(Lava|Sim)$) and the Negative Predictive Value ($\text{Pr}(\neg Lava|\neg Sim)$) are then used to further charactarize model performance against the 2012-3 Tolbachik lava flow. These metrics can provide insight into improving model performance and decision making in volcanic crises.

\section{Introduction}
	Lava flows as a gravity current on the surface of the Earth when liquid magma is effused at the surface with little or no explosivity. In the vacinity of active volcanoes, lava flows represent significant long term impact to infrastructure \citep{peterson2000lava}. In the past, lava flow hazard has been mitigated with the construction of physical diversions and at least once in 252 A.D. by the supernatural grace of St. Agatha of Sicily who died the year prior. Modern science suggests, however, that forecasting the flow path of lava from active volcanoes might be more useful than St. Agatha for communities impacted by effusive volcanism.

	Methods of forecasting lava flows range from simple predictions using empirical relationships between magma flux and flow length \citep{Glaze2003}, to 1-D numerical solutions such as FLOWGO \citep{harris2001flowgo}, to advanced computational fluid dynamics codes like lavaSIM \citep{hidaka2005vtfs}. All modern numerical flow models by nature trade precision in simulating physical processes with computer run-time, so that while FLOWGO is relatively fast it only predicts downslope flow length, while lavaSIM solves Navier-Stokes equations to produce a 3-D flow distribution at the expense of large computational requirements.
	
	Cellular Automata (CA) methods \citep{wolfram1984cellular} have been developed to simulate fluid flow, including lava spreading \citep{barca1994cellular}. In contrast to CFD codes, these do not generally attempt to compute Navier-Stokes equations but instead abstract many physical parameters, such as viscosity and temperature, into more or less empirical rules. The benefit of CA methods for simulating lava flows is most noticeable in the reduced computer time necessary for simulation compared to CFD methods.
	
	Multiple CA lava flow algorithms exist, such as SCIARA \citep{crisci2004simulation}, MAGFLOW \citep{del2008simulations}, ELFM \citep{damiani2006lava}, and LavaPL \citep{connor2012probabilistic}. These algorithms are variations on a theme, where the largest difference between each is how lava is distributed from one automaton to its neighbors. For instance, three versions of SCIARA allow for lava to spread in cardinal directions \citep{barca1994cellular}, in hexagonal directions \citep{crisci2008lava}, or in directions based on an inherent velocity calculated in an eulerian way for each automaton \citep{avolio2006sciara}. MAGFLOW and ELFM by contrast to the original SCIARA algorithm implements 8 directions of spreading. LavaPL and SCIARA both spread in four directions but the apportionment of lava from one automaton to neighbors is based on a different algorithm.

	While several lava flow simulators now exist, each have been made and tested with different lava flows or aspects of flows in mind. Because of this, selecting a specific algorithm to effectively model lava flow hazards can be a necessary, if unwanted challenge. To address this problem, we propose a hierarchical validation scheme to objectively test different flow spreading algorithms. Tests can be designed with different validation levels in mind, to compare lava flow simulations to increasingly complex models, from simply conserving mass to replicating the paths and ultimate areal extents of real lava flows. The tests described in this project can be applied to any flow algorithm that provides at least a list or map of inundated locations over various topographies.

	In this paper, multiple lava flow algorithms are tested using a new modular lava flow code, which I have named MOLASSES (standing for \textit{MOdular LAva Simulation Software in Earth Science}). This code, implemented in C, is a Cellular Automata code which tracks a population of equal-area spaced cells over a grid, that is defined by a digital elevation model (DEM). These cells may or may not be inundated with lava and they are governed by universal rules. Because MOLASSES has been designed in a modular way, it is relatively quick to modify the flow algorithm. Using this code enables code output in a constant format, despite changing methods of lava distribution, which simplifies the comparison of methods.
	
	In Section \ref{sec:MOLASSES}, I will demonstrate how CA is applied to lava flows and detail how a CA simulation is carried out in the MOLASSES code. I will introduce a validation hierarchy in Section \ref{sec_validationb} that can be used to verify and validate different lava flow algorithms using increasingly complex model parameters. In Section \ref{sec:Bayesian}, I will expand on the final validation level (validation against real lava flows) with a Bayesian approach to improving model performance for the 2012-3 Tolbachik Lava Flow. The results from these Sections will be discussed in Section \ref{sec:discussion}.
	
	\subsection{Case Study Area: 2012-3 Tolbachik Lava Flow}\label{sec_tolb_back}
	In the third validation level (Section \ref{sec_tolb_bench}), the recent lava flows at Tolbachik will be used as an example validation test to evaluate flow algorithms against a real lava flow. These flows will then be used in Section \ref{sec:Bayesian} as examples of how a Bayesian approach to evaluating model performance can improve model performance. 
	
\begin{figure}[!h]
	\centering
	\includegraphics[width=0.7\linewidth]{\FigPath/locator_complete_300dpi}
	\caption[The Tolbachik region of Kamchackta, Russia]{The Tolbachik region of Kamchackta, Russia. The two main vents are shown as triangles and the outline of western lava flows, emplaced in 2014, is drawn in red.}
	\label{fig_locator}
\end{figure}
	
The Tolbachik lava flow began in November 2012, originally being sourced from a long fissure vent south of Tolbachik Dol. Initial magma flux was estimated to be 440~m$^3$~s$^{-1}$ \citep{belousov2015overview}. The fissure vent ultimately coalesced into two main vents, seen in TanDEM-X interferometric synthetic aperture radar (InSAR) data \citep{kubanek2015lava}, and the flux dropped significantly to between 100 and 200~m$^3$~s$^{-1}$. Early stages of the flow carried lava west to a maximum runout of 14.5~km and later stages beginning in January or February, carried lava east. The total emplacement volume is $\sim$0.53$\pm$0.07~km$^3$ with 0.38~km$^3$ of that being to the west. TanDEM-X InSAR data has been used to show that the modal thickness of the flow is 7.8~m, and that the overall thickness distribution is log-normal \citep{kubanek2015lava}. After the flow ceased, the total emplacement area was mapped using orthophotos and TanDEM-X data where clouds were present in the images by \citet{kubanek2015lava}.

Figure \ref{fig_locator} shows the outline of the early lava flows, which traveled from two vents along a fissure to the west. This areal extent will be used to validate lava flow simulators. The flow volume is taken to be the total emplacement area within this outline, 26~km$^2$, multiplied by the observed modal thickness of the flow, 7.8~m. The total flow volume used the input parameter in the flow simulations will be 0.22~km$^3$. The remainder of the total emplacement volume to the west of the vents, 0.16~km$^3$ is interpretted to be material that built near-vent edifices (e.g. cones) \citep{kubanek2015lava}. The volume interpretted to be emminated from the northern vent is $4.63\times 10^7$~km$^3$, while the southern vent volume is $1.74\times 10^8$~km$^3$. This estimate was made by splitting the flow between areas north of the Menyailov (northern) vent and south of it, assuming that flows from the Menyailov vents traveled north.
	

%%%%%%%%%%%%%%%%%%%%%%%%%%%%%%%%%%%%%%%%%%%%%%%%%
%MOLASSES%%%%%%%%%%%%%%%%%%%%%%%%%%%%%%%%%%%%%%%%
%%%%%%%%%%%%%%%%%%%%%%%%%%%%%%%%%%%%%%%%%%%%%%%%%
%%%%%%%%%%%%%%%%%%%%%%%%%%%%%%%%%%%%%%%%%%%%%%%%%
%%%%%%%%%%%%%%%%%%%%%%%%%%%%%%%%%%%%%%%%%%%%%%%%%

\section{A Modular Cellular Automata Algorithm for lava flows}\label{sec:MOLASSES}

CA in lava flows has historically been defined as a 2-dimensional space, which is divided into equal-area grid cells, such as those found in a common digital elevation model (DEM). Within the location of each cell is defined an ``elementary automaton'' (\textit{ea}) that has a set of properties, is governed by a set of global rules, and has a set list of neighboring automata. While the behavior rules that each \textit{ea} follow is identical to those of all other automata, its behavior is only dictated by local phenomena. Specifically, the amount of lava that flows in or out of an \textit{ea} will depend on properties such as lava thickness and elevation within it and its neighbors. Because grid cells and \textit{ea} are fundamentally inseperable in this application, I will refer to \text{ea} as cells.

The set of cellular automata is defined as
	\begin{equation}
		\mathbf{A} = \mathrm{\{E^2, V, S, X, \sigma, \gamma\}}
	\end{equation}
	where E$^2$ is the set of point locations of cells in \textbf{A}, V$\subset$E$^2$ is the set of vent or source locations, S is the set of substates within each cell, and X is the local neighborhood that each cell can directly influence \citep{barca1994cellular}. $\sigma$ and $\gamma$ represent the transition functions and source functions within \textbf{A}. 
	
	Practically, E$^2$ is a set of coordinate pairs denoting row and column addresses of cells in a larger grid. S($i$,$j$), which represents the set of substates for the cell at row $i$, column $j$, includes S$_e$, the underlying elevation of an automaton; S$_h$, the thickness of lava within the cell; and S$_{h0}$, the critical thickness, above which lava will spread from a cell. Some algorithms include S$_T$, or the cell temperature in this set. X, in a four-connected neighborhood scheme, is given as \{(0,1), (0,-1), (1,0), (-1,0)\}, where (0,0) is the location of a cell under evaluation. $\sigma$ is the change of substates in S for each cell from timestep $t$ to $t+1$, or S$^{t}\rightarrow$S$^{t+1}$. $\gamma$ specifies the lava emitted at locations within V. The implementation of these sets within the CA structure \textbf{A} is described in detail below.
	
	\subsection{MOLASSES Algorithm Outline}
		MOLASSES is a Cellular Automata code developed in the C programming language based on the CA algorithm ``LavaPL'' of \citet{connor2012probabilistic}. The major change between LavaPL and MOLASSES is that MOLASSES is constructed with nine modules that each have a specific task, either carrying out the CA simulation, reading model input, or writing model output (Figure \ref{fig_flowchart}). The nine modules were designed to replicated major functions in LavaPL and are:
		\begin{enumerate}
			\item{\textbf{DRIVER}} Calls modules in sequence to execute the flow algorithm.
			\item{\textbf{INITIALIZE}} Reads a user-provided configuration file to define model parameters.
			\item{\textbf{DEM\_LOADER}} Imports a raster file to define the elevation model.
			\item{\textbf{INITFLOW}} Uses model parameters to define data arrays.
			\item{\textbf{PULSE}} Incrementally adds lava to source locations.
			\item{\textbf{DISTRIBUTE}} Determines whether to spread and how to spread lava between cells.
			\item{\textbf{NEIGHBOR\_ID}} Identifies the cell neighborhood.
			\item{\textbf{ACTIVATE}} Adds newly inundated cells to the list of active cells.
			\item{\textbf{OUTPUT}} Writes model results to a file using user-specified formats.
		\end{enumerate}
		
		\begin{figure}[!h]
			\centering
			\includegraphics[width=0.3\linewidth]{\FigPath/Flow_Chart}
			\caption[MOLASSES flow chart]{A flow chart of MOLASSES carried out within the \textbf{DRIVER} module. Gray boxes denote various modules, with major inputs and outputs given above and below. Parallelograms are checks performed within DRIVER itself. Rounded boxes represent external input and output files.}
			\label{fig_flowchart}
		\end{figure}
		
		
		Like LavaPL, model parameters are specified by a user through a text configuration file, which must include 1) a digital elevation model (DEM), 2) a residual lava flow thickness, 3) at least one vent location, 4) the total volume and ``pulse volume'' of this vent, and 5) an output file path. The lava flow thickness defines the CA value of S$_{h0}$, where cells with flow thicknesses S$_h>$S$_{h0}$ will spread all lava to their neighboring cells, while cells with less lava will retain their lava. The ``pulse volume'' defines $\gamma$ and the amount of lava to emit at the source location at each time step. The total volume constrains $\gamma$ as lava will not be introduced to the source location after the total volume has been delivered. Modules within MOLASSES that further execute the CA simulation are detailed below.
		
	\subsection[Cells in E2]{Cells in E$^2$}
		%DEM_LOADER
		%INITFLOW
		%ACTIVATE
		Information for cells in the grid defined by E$^2$ is stored in two ways, for code efficiency. First, some information of the CA structure \textbf{A} is stored in a Global Data Grid. This grid stores information known at the beginning of the simulation, such as the user supplied residual flow thickness and the elevation. Grid dimensions are set in the \textbf{DEM\_LOADER} module to be identical to the user-specified raster DEM. This module then imports the elevation of each raster pixel into the corresponding grid cell location. After this operation, the residual flow thickness is also stored in the grid.

		The second information storage method is a list defined in the \textbf{INITFLOW} module. The ``Active List'' is declared with a length that corresponds to the theoretical maximum number of cells that can be inundated by lava. This list contains data that is updated during the simulation, including lava thicknesses, $S_h$, within cells. As cells are determined within the simulation to be inundated with lava for the first time, their row and column addresses, as well as their lava thicknesses are appended to the Active List with the module \textbf{ACTIVATE}.
		
	\subsection[Source Locations and the Source Function]{Source Locations, V, and the Source Function, $\gamma$}
		%INITFLOW
		%PULSE
		Initially in the Active List, \textbf{INITFLOW} only declares source location(s) as the first few elements of the list. These source locations are flagged in the list to be identified as source locations by other modules.

		The \textbf{PULSE} module carries out the source function, $\gamma$. In this module, a separate array stores each source vent's volume parameters. The pulse volume is added to the quantity of lava in the source cell and is subtracted from the remaining volume. The remaining volume is initially set as the total volume given in the configuration file, so PULSE continues to add lava to the source locations at each time step until remaining volume is 0.
		
	\subsection[Substates and the Transition Function]{Substates, S, and the Transition Function, $\sigma$}
		%INITFLOW
		%DISTRIBUTE
		Substates which cannot change, such as the cell elevation S$_e$ and the residual flow thickness S$_{h0}$, are stored within the Global Data Grid. Substates which do change, primarily flow thickness, S$_h$, are stored in the Active List and are allowed to change from timestep to timestep. These values are initialized in \textbf{INITFLOW} where thicknesses are set to 0.
		
		The transition function, $\sigma$, is defined in the \textbf{DISTRIBUTE} module. Cells in this module are evaluated in order of their inundation (i.e. vents are evaluated first and distal cells are evaluated last). The incoming and outgoing quantity of lava from each cell is stored in the Active List. Generally, if a cell has a flow thicknesses S$_h>$S$_{h0}$, it will spread the lava above S$_{h0}$ to any neighbors lower in elevation than itself. When all inundated cells have been evaluated, the incoming and outgoing quantities of lava of each cell are applied to the cells. This flow transition represents a timestep as all cells are updated at once.
		
		Multiple possible transition functions can effectively spread lava from and to cells in a manner that might replicate lava in real life. Identifying transition functions that spread lava in a realistic way is the purpose of the validation tests described in Section \ref{sec_validationb}. In this project three main variations are combined and tested which vary 1) how local slope affects spreading, 2) the neighborhood size, and 3) if any neighbors are eliminated from the neighborhood based on their relationship to the cell.
		
		\begin{figure}[!h]
			\centering
			\includegraphics[width=0.5\linewidth]{\FigPath/slope-proportional-example}
			\caption[A 2-D example of two transition functions with different slope treatments]{A 2-D example of two transition functions. At timestep $t$ (top), Two cells are inundated with lava. The central cell (Cell 0) has 1 block of lava higher than the residual thickness, S$_{h0}$. In a slope-proportional sharing scheme, timestep $t+1$ will follow the path to the right; because Cell -1 has twice the relief as Cell 1, it receives twice as much of the residual lava (2/3 blocks vs. 1/3 to the right). In an equal-sharing scheme, the left path will be followed, and half the block will be added to both neighbor cells.}
			\label{fig_BernieSanders}
		\end{figure}
		
		\paragraph{Local slope-based spreading} In the LavaPL algorithm given by \citet{connor2012probabilistic}, lava is apportioned from cells to their neighboring cells proportional to slope. To give a specific case, let a cell at location $c$ be the central cell, with a set of neighbor cells, X. The total relief between cell $c$ and its lower neighboring cells is 
		\begin{equation}
			\text{TR}(c) = \sum^{n\in \text{X}}(\text{S}_h(c)+\text{S}_e(c))-(\text{S}_h(n)+\text{S}_e(n))
			\label{eq_TR}
		\end{equation}
		where $\text{S}_h$ is the height or thickness of the lava in a cell, $\text{S}_e$ is the underlying elevation of the cell, and $n$ is a neighbor in X. The total lava to spread away from the central cell is the difference between thickness of lava (S$_h$) at $c$ the residual thickness (S$_{h0}$), unless the lava thickness is lower than the residual thickness, giving
		\begin{equation}
			\[ \text{Outbound}(c) =
			\begin{cases}
			\text{S}_h(c)-\text{S}_{h0}(c) & \quad \text{if } {S}_h(c)-\text{S}_{h0}(c) > 0\\
			0 & \quad \text{if } {S}_h(c)-\text{S}_{h0}(c) \le 0\\
			\end{cases}
			\]
			\label{eq_excess}
		\end{equation}
		
		In LavaPL, the excess flow, ``Outbound'', is delivered to neighbors $n$ based on the proportion of total relief, TR, found at each neighbor location (the right path of Figure \ref{fig_BernieSanders}). For each $n\in$X,
		\begin{equation}
			\text{Inbound}(n) = \text{Outbound}(c)\left(\frac{(\text{S}_h(c)+\text{S}_e(c))-(\text{S}_h(n)+\text{S}_e(n))}{\text{TR}}\right)
			\label{eq_propshare}
		\end{equation}
		This is the slope-proportional spreading equation. Another method would be ``slope-blind,'' and would spread lava to all lower neighbors equally following the equation
		\begin{equation}
			\text{Inbound}(n) = \left(\frac{\text{Outbound}(c)}{|\text{X}|}\right)
			\label{eq_equalshare}
		\end{equation}
		where $|\text{X}|$ is the size of the neighborhood, or the number of elements in the neighborhood. This is illustrated as the left path of Figure \ref{fig_BernieSanders}.
		
		\paragraph{Neighborhood size} The size of the neighborhood, X, in CA algorithms is commonly 4 or 8 in cardinal or ordinal directions. Here both have been implemented and both 4- and 8- connected neighborhoods are evaluated in validation exercises later. Neighborhood size is further described in the next section (\ref{sec_X}).

		\paragraph{Spreading inhibited by special relationships} Though the size of the neighborhood is set globally for all cells, neighbors are not guaranteed to receive lava from central cells. In all algorithms, for example, cells in the neighborhood that are higher than the central cell, including lava thicknesses, are excluded from the neighborhood set.
		
		\begin{figure}[!h]
			\centering
			\includegraphics[width=0.5\linewidth]{\FigPath/parent-child-example}
			\caption[A 2-D example of different transition functions with different ``parentage'' rules]{Another 2-D example of different transition functions. In the first timestep (top), Cell -1 initially inundates Cell 0, creating the Parent-Child relationship shown in the next illustrated timestep (middle). If Parents cannot receive lava from Child cells, all residual lava in Cell 0 will flow to Cell 1, following the path to the right. If these relationships are ignored, as shown in the left path, Cell 0 will spread lava in both directions.}
			\label{fig_ParentTrap}
		\end{figure}
		
		Other neighbor elimination rules can also be implemented. One has been designed by \citet{connor2012probabilistic}, where the cell that initially gives lava to another cell is forever eliminated from the receiving cell's neighborhood. This is done by creating a ``parent-child'' relationship for each activated cell in the flow. Simply put, child cells cannot give lava to their parent cells (right path in Figure \ref{fig_ParentTrap}). This transition function rule is tested against no parentage rules in competing MOLASSES algorithms (left path in Figure \ref{fig_ParentTrap}).
		
	\subsection{Cell Neighborhood, X}\label{sec_X}
	
	\begin{figure}[!h]
		\centering
		\includegraphics[width=0.5\linewidth]{\FigPath/neighborhoods}
		\caption[Cellular Automata neighborhoods]{Cellular Automata neighborhoods. To the left, in a 4-connected neighborhood, a central cell may influence or be influenced by cells in cardinal directions. To the right, in an 8-connected neighborhood, the zone of influence is expanded to include ordinal directions. Numbers in each cell are relative weights (determined by distance from the central cell), so diagonal neighbors are weighted less than orthogonal cells.}
		\label{fig_MrRogers}
	\end{figure}
	
		%NEIGHBOR_ID
		The final set in the CA is the cell neighborhood X and is defined by the \textbf{NEIGHBOR\_ID} module. This neighborhood is usually either 4-connected (von Neumann neighborhood) or 8-connected (Moore neighborhood) as illustrated in Figure \ref{fig_MrRogers}. Four-connected neighborhoods are defined as the row, column coordinates \{(0,1), (0,-1), (1,0), (-1,0)\}, where (0,0) is the location of a cell under evaluation, while the set elements might correspond to North, South, East, and West. Eight-connected neighbors include the ordinal directions, Northeast, Southeast, Northwest, and Southwest: \{(0,1), (0,-1), (1,0), (-1,0), (1,1), (-1,-1), (1,-1), (-1,-1)\}.
		
		NEIGHBOR\_ID is implemented within the DISTRIBUTE module to evaluate cells within X, and determine whether they are lower in elevation (including their lava) than the central cell. If one is lower, NEIGHBOR\_ID returns their relief, or the difference in elevation between the cell and the central cell, to the DISTRIBUTE module. Depending on whether parent-child relationships are recorded or ignored in the transition function, NEIGHBOR\_ID can follow one of two algorithms below.
		\begin{center}
		\begin{tabular}{l}
			\toprule
			\textbf{4-connected NEIGHBOR\_ID}\\
			\textbf{module}\\
			\midrule
			X = \{(0,1), (0,-1), (1,0), (-1,0)\}\\\\
			X' = \{\}\\
			$c$ = (0,0)\qquad (central cell location)\\
			\textbf{For} $n\in$ X\\
			~~\textbf{If}~$(\text{S}_h(c)+\text{S}_e(c))-(\text{S}_h(n)+\text{S}_e(n)) > 0$\\
			~~~~\textbf{Append}~$n$ to X'\\\\
			\textbf{Return}~X'\\
			\bottomrule
		\end{tabular}
		\begin{tabular}{l}
			\toprule
			\textbf{8-connected NEIGHBOR\_ID}\\
			\textbf{with Parent-Child Relationships}\\
			\midrule
			X = \{(0,1), (0,-1), (1,0), (-1,0), \\
			\qquad~(1,1), (-1,-1), (1,-1), (-1,-1)\}\\
			X' = \{\}\\
			$c$ = (0,0)\qquad (central cell location)\\
			\textbf{For} $n\in$ X\\
			~~\textbf{If}~$(\text{S}_h(c)+\text{S}_e(c))-(\text{S}_h(n)+\text{S}_e(n)) > 0$\\
			~~~~\textbf{If} $n$ is \textbf{not} Parent of $c$\\
			~~~~~~\textbf{Append}~$n$ to X'\\
			\textbf{Return}~X'\\
			\bottomrule
		\end{tabular}
		\end{center}


			
			
			
%%%%%%%%%%%%%%%%%%%%%%%%%%%%%%%%%%%%%%%%%%%%%%%%%%
%HIERARCHY
%%%%%%%%%%%%%%%%%%%%%%%%%%%%%%%%%%%%%%%%%%%%%%%%%%

	\section{Validation Hierarchy}\label{sec_validationb}
	
	The validation strategy implemented in this paper follows the advice of \citet{bayarri2007framework} for validating computer models, namely ``1) defining the problem; 2) establishing evaluation criteria; 3) designing experiments; 4) approximating computer model output; 5) analyzing the combination of field and computer run data.'' The sixth step in their validation process, feeding results back to revise models, has been done informally to determine how to alter spreading algorithms in the future. Each level below presents a problem for a lava spreading algorithm to complete. These fundamental problems (e.g. replicating a Bingham flow) are evaluated using simple tests that demonstrate the problem. The relevant model output for each of these tests is a list of locations (i.e. a list of x and y coordinates) that have been inundated by lava. After verification (Level 0), the first validation level tests model results with other model results; the second level tests model output against expected analytical solutions; and the third level tests model output from field data. 
	
	Multiple flow algorithms can pass all of these tests, illustrating that they are valid under certain conditions and can be relied on. Choosing between algorithms which have been validated is based on the needs of the user, but algorithms that do not perform well in these validation tests might not be reliable for other applications.

	\begin{center}
		\begin{table}[h]
		\caption{Transition Algorithm Codes and Descriptions}
		\begin{tabular}{l c p{5cm} p{5cm}}
			\toprule
			Transition&Neighborhood&Parent-Child&Slope-proportional\\
			Function&&Relationships Preserved?&Sharing?\\
			\midrule
			\textbf{4/P/S} &4-directions & Yes, ``parents'' do not accept lava from ``children.'' & Yes, lower cells receive lava based on relative relief.\\
			\textbf{8/P/S} &8-directions & Yes & Yes\\
			\textbf{4/N/S} &4-directions & No, ``parents'' are not defined. & Yes\\
			\textbf{8/N/S} &8-directions & No  & Yes\\
			\textbf{4/P/E} &4-directions & Yes & No, all lower cells receive equal quantities of lava.\\
			\textbf{8/P/E} &8-directions & Yes & No\\
			\textbf{4/N/E} &4-directions & No  & No\\
			\textbf{8/N/E} &8-directions & No  & No\\
			
			\bottomrule
		\end{tabular}
		\label{tab_algorithmcodes}
		\end{table}
	\end{center}

\paragraph{Test Algorithms} Combining three variations of the Transition Function described in Section \ref{sec:MOLASSES}, I have created eight MOLASSES lava flow algorithms. Each variation has been made by modifying one module in the MOLASSES framework: The neighborhood is changed between 4- and 8- directions using the NEIGHBOR\_ID module, classifying one cell as a ``parent'' cell when a location is initially inundated is within the ACTIVATE module, and dividing lava amongst neighboring cells proportional to slope or equally is carried out in the DISTRIBUTE module. These eight algorithms will be referred to using three character codes, listed in Table \ref{tab_algorithmcodes}. For the algorithm used by LavaPL in \citet{connor2012probabilistic}, the code would therefore be 4/P/S, as it spreads lava in 4-directions from a central cell, all inundated cells have designated parents to whom they cannot spread lava, and the quantity of lava to spread from a central cell is higher for lower neighboring cells.

	\subsection{Level 0: Conservation of Mass}
			Before the results of a lava flow simulation can be validated, it must be verified to at least prove that conservation of mass is preserved. A lava flow simulation will therefore not be tested against the following tests until this conservation of mass requirement is shown to be fulfilled.
			
			In MOLASSES, the code is verified within the DRIVER module, which manages each subordinate module. The erupted volume, $V_{in}$, is given as the total eruption volume specified by the user in the configuration file. If multiple source locations are given in this file, $V_{in}$ is the sum of total eruption volumes. $V_{in}$ is compared at the end of the module to the total volume of the flow, or $V_{out}$. The volume $V_{out}$ is calculated by summing the volume in all inundated grid cells. MOLASSES reports success if $V_{in}-V_{out} \le 10^{-8}$~m$^3$, which is the precision of a 64-bit double. If this test fails, MOLASSES reports failure and the excess volume found in the flow.
	
			\begin{center}
				\begin{tabular}{l}
					\toprule
					\textbf{MOLASSES Conservation of Mass Test}\\
					\midrule
					\textbf{If}~$|V_{in}-V_{out}| \le 10^{-8}$\\
					~~\textbf{Print}~\verb|SUCCESS: MASS CONSERVED|\\
					\textbf{Else}\\
					~~$excess = V_{out}-V_{in}$\\
					~~\textbf{Print}~\verb|ERROR: MASS NOT CONSERVED! Excess: |$excess$ \verb|m^3|\\
					\bottomrule
				\end{tabular}
			\end{center}

	\subsection{Level 1: Repeatability given meaningless parameter variation}
		Once the code has been verified to conserve mass, the flow can be validated. This first validation level tests that lava flow simulations are repeatable, regardless of changes in parameter space that should have no effect on the flow. Parameters that ideally should not effect lava flows include slope direction and elevation model resolution. For instance, a slope to the west and an identically dipping slope to the east should produce lava flows of equal length and shape (given identical flow attributes).
		
		\citet{miyamoto1997simulating} performed a simple validation test on two CA-like flow simulators \citep{ishihara1990numerical,miyamoto1997simulating} where a sloped DEM was rotated 45 degrees from ``south'' to ``southeast''. This test was performed to demonstrate that the flow models had the same run-out length regardless of the arbitrary slope direction. Here, the DEM rotation scheme by \citet{miyamoto1997simulating} is adopted and expanded, so that a DEM of a simple slope is rotated 19 times at increments of 5$^{\circ}$. Flows are simulated on each of these slopes and the locations of inundated cells are output from the model.
		
		Three characteristics of the simulated flows are determined for each slope direction: flow length, orientation, and aspect ratio. Flow length is defined as the distance between the vent and the furthest inundated point from the vent. Flow orientation is defined as the direction that furthest point lies, with respect to North. Flow aspect ratio is the ratio of maximum flow width to flow length. Perfect success for this exercise is when simulated flows, regardless of slope direction, 1) do not change in length, 2) have an orientation identical to the slope direction, and 3) do not change in aspect ratio. Failure is more subjective, but I will define failure as 1) more than 10\% variation in flow length depending on slope direction, 2) more than 5$^{\circ}$ offset between the slope and the flow orientation on average, or 3) more than 15\% variation in flow aspect ratio.
		
		\begin{figure}[h!]
			\centering
			\includegraphics[width=\linewidth]{\FigPath/lava_C_4N_slope}
			\caption[Rotating slope test for the LavaPL algorithm]{Rotating slope test for algorithm \textbf{4/P/S} (LavaPL). Slope dip is 18$^{\circ}$, with dip-directions 0N, 30N, and 80N from left to right. The flow length and aspect ratio are similar and the flow direction is in the slope direction, so it passes Level 1 criteria.}
			\label{fig:slope}
		\end{figure}
		
		\subsubsection{Exercise Parameters} The underlying DEM for this exercise has a simple 18$^{\circ}$ slope, dipping to the North. The DEM has a spatial resolution of 1~m. The source cell is placed at the center of the DEM, is given a total volume of 1000~m$^3$, and is given a pulse volume of 1~m$^3$. When the simulation is finished, model output is used to determing the three flow characteristics used in this test (length, orientation, and aspect ratio). The DEM is rotated 5$^{\circ}$ clockwise and the process is repeated 19 times until the flow is simulated on an East-facing slope.
		
		
		
		\subsubsection{Results}

		For all eight flow algorithms, flow length, aspect ratio, and orientation were calculated 19 times, corresponding to the 19 dip directions sampled between 0$^{\circ}$N and 90$^{\circ}$N. Variance for length and aspect ratio were calculated as the ratio of their standard deviations to their means. For instance, if mean runout length for the 18 flows is 100~m and the standard deviation of the 18 lengths is 2~m, the runout length variance is 2\%. The mean direction error is also calculated for the set of flows from each algorithm. These are reported in the table below.

			\begin{center}
				\textbf{DEM Rotation Results}
				
				\begin{tabular}{l c c c}
					\toprule
					Transition&Run-out&Aspect Ratio&Mean Direction\\
					Function&Variance&Variance&Error\\
					\midrule
					4/P/S &2.7\%&6.7\%&1.2$^{\circ}$\\
					8/P/S &4.4&12.2&0.9\\
					4/N/S &9.6&19.7&1.3\\
					8/N/S &3.9&7.5&0.6\\
					4/P/E &21.6&38.6&14.2\\
					8/P/E &7.2&13.8&5.4\\
					4/N/E &21.6&38.7&14.1\\
					8/N/E &7.2&13.8&5.5\\
			
					\bottomrule
				\end{tabular}
			\end{center}

			While with an ideal spreading algorithm, variances and direction error would be 0 under a rotating slope, every spreading algorithm tested performed differently as DEM direction changed. Following from the above pass-fail standards, five of the eight algorithms can be rejected. Algorithms 4/P/E and 4/N/E have high run-out length variance. Algorithms 4/N/S, 4/P/E and 4/N/E have large aspect-ratio variance. Algorithms 4/P/E, 8/P/E, 4/N/E, and 8/N/E all systematically deviate from running downslope by $>5^{\circ}$ on average. This implies that algorithms which share lavas equally from central cells to all lower neighboring cells perform worse than algorithms which share lavas proportional to slope.
			
			For the eight different transition functions tested, runout length varied between 60-160~m. The flow algorithm with the least flow length variance was the 4-connected, parent-child, slope-proportional strategy implemented in LavaPL. Algorithms 4/P/S (LavaPL), 8/P/S, and 8/N/S cannot be rejected because of any of the three standards set in this exercise.

	\subsection{Level 2: Replication of flow morphologies on simple physical surfaces}
	
	The second validation level is the first step in validating lava flow algorithms against realistic flow expectations. Instead of parameter space being arbitrarily defined, which was the case in Level 1, the defined parameter space informs tests at this level as to what the model output should be. As lava flows on a large scale are well described as Bingham fluids, simulations can be tested against analytical solutions or experimental observations of these fluids in simple conditions. For instance, a lava flow on a perfectly flat surface might be expected to create a circular areal extent \citep{griffiths2000dynamics}.
	
			Here I measure flow algorithm performance on a flat surface from a single vent source location. To measure the extent to which the simulated flow replicates a circle, the inundated area is compared to the area of a circle which circumscribes the flow exactly. This can be described as
			\begin{equation}
				Fit = \frac{A_{flow}}{\pi d_{max}^2}
			\end{equation}
			where $d_{max}$ is the farthest extent of the simulated flow from the vent. A perfect match to a circle would result in a $Fit=1$. With the same maximum distance from the vent (i.e. the distance from the center to a vertex) a perfect square would cover 64\% of the area of a circle, ergo $Fit=0.64$. An octagon would have a fit of 0.90. We consider a model to successfully pass this test if it produces a flow of $Fit>0.90$, or if the flow approximates a circle better than an octagon. The model unambiguously fails this test if if produces a flow of $Fit<0.64$, where a square better describes a circle than a flow generated from the model.

			\begin{center}
				\textbf{Circular Flow Test}
				
				\begin{tabular}{l l}
					\toprule
					Fit = 1.0 & Best Possible Score; perfectly circular.\\
					Fit $>$ 0.90 & Success; better than an octagon. \\
					Fit $<$ 0.64 & Failure; worse than a square.\\
					\bottomrule
				\end{tabular}
			\end{center}
		
		\begin{figure}[!h]
		\centering
		\includegraphics[width=\linewidth]{\FigPath/pancake}
		\caption[Fitness scores of different shaped flows on a flat surface]{Fitness scores of different shapes. From left to right: A circle has a perfect fit score of 1.00; an octagon has 0.90 times the area of a circumscribing circle; a square has a score of 0.64; Two flat surface tests for slope proportional spreading algorithms with parent rules. The flow second to the right is 4-connected (4/P/S) and has a score of 0.55, while the rightmost flow is 8-connected (8/P/S) with a score of 0.95. While the 4/P/S flow scores worse than a square, its 8-connected version passes the test as it scores better than an octagon.}
		\label{fig_pancake}
	\end{figure}
	
		\subsubsection{Exercise Parameters} The DEM used in this exercise is a horizontal plane (all grid locations have the same elevation) with a spatial resolution of 1~m. A single vent is located at the DEM center with a total volume of 1000~m$^3$, and a pulse volume of 1~m$^3$. When the simulation is finished, model output is used to find the inundated cell farthest from the vent ($d_{max}$). The total inundated area is divided by the area of a circle with radius $d_{max}$ to provide the Fit score.
		
		\subsubsection{Results}

		Five of the eight algorithms unambiguously passed the test of performing better than an octagon. In this test 8-connected algorithms outperformed 4-connected algorithms. Four algorithms unambiguously passed this test: 8/P/S, 8/N/S, 4/N/E, and 8/N/E.
		
		\begin{center}
			\textbf{Bingham Circle Results}\\
			\begin{tabular}{l c l}
				\toprule
				Algorithm&Circularity&\\
				\midrule
				4/P/S & 0.55 & Worse than a square.\\
				8/P/S & 0.95 & Better than an octagon.\\
				4/N/S & 0.55 & Worse than a square.\\
				8/N/S & 0.98 & Better than an octagon.\\
				4/P/E & 0.77 & Between a square and an octagon.\\
				8/P/E & 0.80 & Between a square and an octagon.\\
				4/N/E & 1.00 & Perfectly circular.\\
				8/N/E & 0.99 & Better than an octagon.\\
				
				\bottomrule
			\end{tabular}
		\end{center}
		
	\subsection{Level 3: Replication of real lava flows over complex topography}\label{sec_tolb_bench}
		The recent availability of global or near-global topographic datasets, such as SRTM or ASTER GDEM has enabled the direct observation of the underlying surface of even more recent lava flows. Flow algorithms can be validated against recent lava flows by simulating lava over these surfaces with parameters defined by the new lava flows. The 2012-3 Tolbachik lava flows will be used as a validation exercise for the eight flow algorithms. As discussed above (Section \ref{sec_tolb_back}), the earliest lavas flowed from a fissure to the west. Before later stage flows began moving to the east, the volume of the lavas were 0.22~km$^3$. The modal thickness of the flow has been found to be 7.8~m and the areal extent was mapped with orthoimages \citep{kubanek2015lava}.

		For this example, two metrics which are commonly employed to validate lava flow simulators against real flows will be used: model sensitivity and a fitness metric called the ``Jaccard coefficient.'' An alternative bayesian approach to these metrics is discussed in Section \ref{sec:Bayesian}. These two metrics fundamentally attempt to quantify the amount of agreement between the lava flow and a simulated flow. This is visualized as a 2x2 matrix (Figure \ref{fig_2x2}), where agreement between a simulation and an actual lava flow (i.e. locations inundated by lava in real life and the simulation) are known as True Positives. Areas inundated by only the simulation are False Positives, while areas inundated by the lava flow that the simulation fails to forecast as being inundated are False Negatives. Locations that neither the flow or simulation inundate are True Negatives. 
		
		
\begin{figure}
			\centering
			\includegraphics[width=0.4\linewidth]{\FigPath/2xtable_300dpi.png}
			\caption[2x2 table comparing regions inundated by real lava and a simulated flow]{2x2 table comparing regions inundated by real lava and a simulated flow. In the bottom left, True Positives are areas engulfed by lava both in real life and in the simulation. In the top left, False Negatives are areas hit by lava where the simulation failed to model inundation. In the bottom right, False Positives are locations where the simulation forecast inundation, but which remained untouched by lava in reality. Finally, in the top right, True Negatives are areas which were not inundated by lava and the simulation successfully forecast their safety.}
			\label{fig_2x2}
		\end{figure}
		
		Model sensitivity is defined as
		\begin{equation}
			\text{Model~Sensitivity}=\frac{|Lava\cap Sim|}{|Lava|}
		\label{eq_sensitivity}
		\end{equation}
		where $|Lava\cap Sim|$ is the areal size of the interesection of the simulation and the true lava flow (i.e. the True Positives) and $|Lava|$ is the areal size of the lava flow (True Positives and False Negatives). This gives a percentage of the true lava flow that the simulation correctly predicted. 
		
		The Jaccard coefficient, or Jaccard fit, also uses True Positives as the numerator, but expands the denominator to include all areas inundated by either the lava flow or the simulated flow. This is defined as 
		\begin{equation}
			\text{Jaccard~Fit}=\frac{|Lava\cap Sim|}{|Lava\cup Sim|}
		\end{equation}
		where $|Lava\cup Sim|$ is the size of the union of the lava flow and a simulated flow (i.e. True Positives, False Negatives, and False Positives). This gives a percentage of the total area covered by simulated flows and/or true flows that is covered by both.
		
		Each flow algorithm is run using the following parameters. Flows are run over both 3-arcsecond SRTM topography (75 m grid resolution at 56$^{\circ}$N) and bistatic TanDEM-X topography processed by \citet{kubanek2015lava}. The pulse volume, the volume added to vent cells at each code loop (i.e. each instance of the PULSE module), is set at the product of the grid-cell area (5600 m$^2$ for the SRTM DEM and 225 m$^2$ for the TanDEM-X DEM) and the residual thickness (7.8~m). Both fitness metrics are calculated with the resulting model output, given as a list of inundated locations. ``Failure'' can be defined here as either metric falling below 50\% for the sake of example.
		
		\begin{center}
			\textbf{Tolbachik Validation Flow Parameters}\\
			\begin{tabular}{l l}
				\toprule
				Elevation Model & 75-m SRTM or 15-m TanDEM-X\\
				Residual Thickness & 7.8~m\\
				Pulse Volumes & 44200 m$^3$ (SRTM) or 1800 m$^3$ (TanDEM-X)\\
				\midrule
				Vent$_N$ Easting & 582800~m (UTM Zone 57)\\
				Vent$_N$ Northing & 6182100~m\\
				Vent$_N$ Total Volume & 4.63$\cdot10^7$~m$^3$\\
				\midrule
				Vent$_S$ Easting & 582475~m\\
				Vent$_S$ Northing & 6180700~m\\
				Vent$_S$ Total Volume & 1.737$\cdot10^8$~m$^3$\\
				\bottomrule
			\end{tabular}
		\end{center}
			
	\subsubsection{Results}
	
	\begin{figure}[!h]
			\centering
			\includegraphics[width=\linewidth]{\FigPath/tolb_lvl3_examples_72dpi}
			\caption[Three transition function algorithms simulating the 2012-3 Tolbachik lava flows]{Three simulation algorithms (4/P/S, 4/N/E, and 8/N/S) applied to two elevation models (SRTM and TanDEM-X) to simulate the 2012-3 Tolbachik lava flows, outlined in black. Lava is emitted from two vent locations marked as blue triangles in the simulation. Diagrams showing the relative True Positives (green), False Positives (orange), and False Negatives (red) are illustrated in the top left of each simulation.}
			\label{fig:tolbachik}
		\end{figure}
	
	Three example algorithms are illustrated in Figure \ref{fig:tolbachik}. One primary observation is that all simulations had a longer run-out length on the finer TanDEM-X grid than on the coarser SRTM grid. Despite this, all flows took the correct major pathways taken by the true lava flow. A small diagram in the top left corner of each map in Figure \ref{fig:tolbachik} shows the true positives, false positives, and false negatives in each simulation. True positives are areas inundated by both flow and simulation, false positives are areas simulated as being inundated but are not mapped as such, and false negatives are areas hit by lava that the simulation failed to forecast. 
		
			The best algorithm and DEM pair were the 8/N/S algorithm over SRTM (top left of Figure \ref{fig:tolbachik}), while this same algorithm performed fairly poorly over TanDEM-X topography. For the SRTM simulation, this algorithm achieved a model sensitivity of 82.8\% and a Jaccard fitness score of 63.1\%. Graphically, sensitivity is calculated as the green area in the Figure \ref{fig:tolbachik} diagram divided by the green and red areas. The Jaccard fitness is the green area divided by the total area of the diagram. Because the Jaccard fitness statistic essentially expands the denominator of model sensitivity, it will never be higher than model sensitivity.
			
			If success and failure are defined by having a fits of greater or less than 50\%, all models tested would pass for the SRTM DEM and about half would pass for the TanDEM-X DEM. All but one model (4/P/E) performed worse on the TanDEM-X DEM. The Jaccard fit and Sensitivity for all models are given below.
					
		\begin{center}
			\textbf{Tolbachik Flow Results}\\
			\begin{tabular}{l c c | c c}
				\toprule
				Transition&\multicolumn{2}{c}{\textbf{SRTM DEM}}&\multicolumn{2}{c}{\textbf{TanDEM-X DEM}}\\
				Function& Jaccard & Sensitivity& Jaccard & Sensitivity\\
				\midrule
				4/P/S & 56.7\%& 76.4\%& 53.0\%& 72.4\% \\
				8/P/S & 61.1  & 80.8  & 46.8  & 67.2   \\
				4/N/S & 57.2  & 77.5  & 44.0  & 64.0   \\
				8/N/S & 63.1  & 82.8  & 46.7  & 67.4   \\
				4/P/E & 51.2  & 71.5  & 54.2  & 73.4   \\
				8/P/E & 58.8  & 78.2  & 56.3  & 76.0   \\
				4/N/E & 54.5  & 74.5  & 55.7  & 73.7   \\
				8/N/E & 59.6  & 78.8  & 56.2  & 75.3   \\
				
				\bottomrule
			\end{tabular}
		\end{center}


%%%%%%%%%%%%%%%%%%%%%%%%%%%%%%%%%%%%%%%%%%%%%%%%
%BAYESIAN APPROACH
%%%%%%%%%%%%%%%%%%%%%%%%%%%%%%%%%%%%%%%%%%%%%%%%
%%%%%%%%%%%%%%%%%%%%%%%%%%%%%%%%%%%%%%%%%%%%%%%%
%%%%%%%%%%%%%%%%%%%%%%%%%%%%%%%%%%%%%%%%%%%%%%%%
\section{Bayesian Applications for Lava Flow Models}\label{sec:Bayesian}
	The final step \citet{bayarri2007framework} give for validating computer models is ``feeding [observations and results] back to revise the model.''
	
	The use of computer models to forecast hazards is a fundamentally Bayesian strategy: there is an initial concern due to hazards and computer models help us inform, constrain, and update this concern. Using Bayesian statistics can therefore be an improvement in testing lava flow models, over the two commonly used fitness tests, model sensitivity and the Jaccard index, because of their more direct application to informing perceived risk. 
	
	Three tools will be used in this section: the Positive Predictive Value, the Negative Predictive Value, and a Bayes factor. Bayes theorem connects a phenomenon $A$ to observations $B$ through the function
	\begin{equation}
		\text{Pr}(A|B)=\frac{\text{Pr}(B|A)\text{Pr}(A)}{\text{Pr}(B)}\label{eq_bayes}
	\end{equation}
	where $\text{Pr}(A)$ is the general probability of $A$ occuring, $\text{Pr}(B)$ is the probability of $B$ being observed, and $\text{Pr}(B|A)$ is the conditional probability of $B$ given the occurence of $A$. $\text{Pr}(B|A)$ is also known as model sensitivity, which is a common fitness statistic and was discussed in Section \ref{sec_tolb_bench}
	
	The left side of Equation \ref{eq_bayes}, $\text{Pr}(A|B)$, is the Posterior probability of $A$ and can be stated as ``the probability that lava will inundate a location if the model forecasted inundation at that location,'' or the Positive Predictive Value (PPV) of the model. A second posterior is the Negative Predictive Value (NPV) and is $\text{Pr}(\neg A|\neg B)$, the obverse of $\text{Pr}(A|B)$. This negative predictive value relates not being inundated by lava at a given location to a safe outcome forecasted by a simulation and can be calculated by modifying Equation \ref{eq_bayes} and substituting $A$ for $Lava$ (the lava flow) and $B$ for $Sim$ (the simulation), resulting in the formula
	\begin{equation}
		\text{Pr}(\neg Lava|\neg Sim)=\frac{\text{Pr}(\neg Sim|\neg Lava)\text{Pr}(\neg Lava)}{\text{Pr}(\neg Sim)}
	\end{equation}
	where the $\neg$ symbol indicates the event or observation not happening, and $\text{Pr}(\neg Sim|\neg Lava)$ is model specificity.
	
	The negative predictive value is important in hazard forecasting as it is in some sense a probability of safety. The more common positive predictive value $\text{Pr}(A|B)$ (or $\text{Pr}(Lava|Sim)$) does not contain information about areas that the simulation does not inundate, as it is essentially the True Positive area divided by the area of simulated inundation (defined later in Equation \ref{eq_simplepost}). While the PPV can support the hypothesis that lava will hit a location given a simulated hit, it cannot estimate one's relative risk if the simulation forecasts a safe outcome. The NPV, $\text{Pr}(\neg Lava|\neg Sim)$, does just this, and informs a user whether to rely on a safe outcome from a simulation. If, for example, the PPV $\text{Pr}(Lava|Sim)$ is high while the NPV $\text{Pr}(\neg Lava|\neg Sim)$ is low for a given lava flow simulator, areas that are evacuated due to the simulation outcome will be evacuated for good reason, but many areas will likely be inundated that were not evacuated due to the simulation outcome. It is important to estimate and ultimately try to improve both predictive values, because it is important to understand both how well a simulation matches areas inundated by lava and how well it matches areas not inundated by lava.
	
	Bayes factors provide a tool to test the relative likelihood of a hypothesis against another. \citet{aspinall2003evidence} introduced this tool to volcano hazard forecasting by testing whether the onset of particular seismic events before the 1993 Galeras catastrophe was a significant indicator of the eruption or not. A Bayes Factor (BF) relating two hypotheses is given by \citet{jeffreys1998theory} as
	\begin{equation}
		\text{BF} = \frac{\text{Pr}(\text{Evidence}|\text{Hypothesis~1})}{\text{Pr}(\text{Evidence}|\text{Hypothesis~2})}
		\label{eq_BF}
	\end{equation}
	In the example of Galeras, the Evidence in Equation \ref{eq_BF} are new seismic events, Hypothesis 1 is ``imminent explosion,'' and Hypothesis 2 is its complement ``not imminent explosion'' \citep{aspinall2003evidence}. Below, I will apply this with the Evidence being the probability of simulated inundation and the Hypotheses ``lava inundation'' and ``not lava inundation.'' Essentially, the Bayes Factor approach identifies whether an area, given evidence provided by flow simulations, is better described as inundated by lava or not inundated by lava. \citet{jeffreys1998theory} provided a log-scale interpretation to the value of BF in Equation \ref{eq_BF}, given in the Table \ref{tab_BFinterps}.
	
	\begin{table}[h]
		\centering
		\caption{Bayes Factor Interpretations (modified from \citet{aspinall2003evidence})}
		\begin{tabular}{l l}
			\toprule
			BF Value & Description\\
			\midrule
			$BF>10^2$ & Evidence for Hypothesis 1 is Decisive.\\
			$10^{1.5}<BF<10^2$ & Evidence for Hypothesis 1 is Very Strong.\\
			$10^{1}<BF<10^{1.5}$ & Evidence for Hypothesis 1 is Strong.\\
			$10^{0.5}<BF<10^{1}$ & Evidence for Hypothesis 1 is Substantial.\\
			$10^{0}<BF<10^{0.5}$ & Evidence for Hypothesis 1 is just worth a mention.\\\\
			$10^{-0.5}<BF<10^{0}$ & Evidence for Hypothesis 2 is just worth a mention.\\
			$10^{-1}<BF<10^{-0.5}$ & Evidence for Hypothesis 2 is Substantial.\\
			$10^{-1.5}<BF<10^{-1}$ & Evidence for Hypothesis 2 is Strong.\\
			$10^{-2}<BF<10^{-1.5}$ & Evidence for Hypothesis 2 is Very Strong.\\
			$BF<10^{-2}$ & Evidence for Hypothesis 2 is Decisive.\\
			\bottomrule
		\end{tabular}
		\label{tab_BFinterps}
	\end{table}
	
		\paragraph{Calculating Predictive Values} In the same manner as the final validation level, the statistics discussed above will be calculated based on the areal extent of flows and simulations. The probability of the lava flow inundating an area $N$ can be given as
		\begin{equation}
			\text{Pr}(A)=\frac{|Lava|}{|N|}\label{eq_PA}
		\end{equation}
		where $|Lava|$ is the areal size of the flow (i.e. literally the number of DEM grid cells the lava inundates) and $|N|$ is the size of the area of interest, or the potential hazard area. The probability of the simulation is similarly found to be
		\begin{equation}
			\text{Pr}(Sim)=\frac{|Sim|}{|N|}.\label{eq_PB}
		\end{equation}
		By substituting these definitions and model sensitivity (Equation \ref{eq_sensitivity}, $|Lava \cap Sim|/|Lava|$) in Equation \ref{eq_bayes}, the posterior probability of lava flow inundation (i.e. the PPV), given a simulation that forecasts inundation can be recast as
		\begin{align}
		\text{Pr}(Lava|Sim)&=\frac{\frac{|Lava\cap Sim|}{|Lava|}\frac{|Lava|}{|N|}}{\frac{|Sim|}{|N|}}~\text{,~or~simplified,}\label{eq_unsimplepost}\\
		&=\frac{|Lava\cap Sim|}{|Sim|}.\label{eq_simplepost}
		\end{align}
		where $|Lava\cap Sim|$ is the size of the intersection of the lava flow and simulated flow (again, the number of DEM grid cells that are True Positives). Note that PPV is independent of the potential hazard area.
		
		the NPV can be stated in terms of the sizes of the lava flow and simulated flow as well.
		\begin{equation}
			\text{Pr}(\neg Lava|\neg Sim)=\frac{|\neg Lava\cap \neg Sim|}{|\neg Sim|}\label{eq_simplenegpost}
		\end{equation}
		In this equation the numerator is the total area of True Negatives, where neither real flows or simulated flows reached. The denominator is the area not hit by the simulated flows, or the True and False Negatives. Calculating the size or number of grid cells of $\neg Lava$ or $\neg Sim$ is fundamentally dependent on the potential hazard area, as $|\neg Lava|$ is defined as
		\begin{equation}
			|\neg Lava| = |N| - |Lava|.
		\end{equation}
		Because of this, we must define the size of the potential hazard area $N$ ($|N|$).	
	
	\paragraph{Potential Hazard Area} There are multiple strategies to estimating an \textit{a priori} hazard area. \citet{kauahikaua1995applications} for instance identified catchments or ``lava sheds'' in which a volcanic vent was erupting, and identified these lava sheds as the hazard area. \citet{kilburn2000lava} provided a theoretical maximum distance that a lava flow can travel given the mass flux of magma erupting at the vent location. A combination of these two would provide an objective hazard area defined as the area within the ``Kilburn distance'' that is topographically below the volcanic vent. The theoretical maximum distance, or hazard radius, given by \citet{kilburn2000lava} is
		\begin{equation}
		R_{max}=\sqrt{\frac{3\epsilon SQ}{\rho g\kappa}}
		\label{eq_kilburn}
		\end{equation}
	where $\epsilon$ is an empirical value related to the amount of extension of lava crust allowed before it fails (10$^{-3}$), $S$ is the tensile strength of this crust (10$^7$~Pa), $\rho$ is the lava crust density (2200~kg~m$^{-3}$), $g$ is gravitational acceleration, $\kappa$ is the bulk thermal diffusivity ($4\times 10^{-7}$~m$^{2}$~s$^{-1}$) and $Q$ is the mean volumetric flow rate from the vent. From this, the hazard radius for the Tolbachik 2012-3 flow is calculated to be 39~km, given a magma flux of 440~m$^3$~s from the vent as was estimated early in the eruption \citep{belousov2015overview}. The total area within this radius that is also below the vent-plus-modal-flow-thickness elevation is 1,415~km$^2$. Note that the mapped flow area of 26~km$^2$ only covers 1.9\% of this defined hazard area (i.e. $\text{Pr}(Lava) = 0.019$).
	
		\begin{figure}
			\centering
			\includegraphics[width=0.7\linewidth]{\FigPath/hazard_areas_72dpi}
			\caption[Potential hazard areas of the 2012-3 Tolbachik Flows]{Potential hazard areas of the 2012-3 Tolbachik Flows in light green, defined by a maximum flow radius (left, Equation \ref{eq_kilburn}) and the total areal coverage of 100,000 random flow simulations (top-right, Section \ref{sec_MC}). The mapped lava flow is red. The chart to the bottom right shows the 2x2 grid illustrated in Figure \ref{fig_2x2}.}
			\label{fig_hazardarea}
		\end{figure}
	
	A second strategy would be to run many lava flow models from the known vent location(s) while varying input parameters. This would give a range of flows and the true flow might be completely contained within the region given by this range of simulations. Below, a Monte Carlo (MC) method will be used to simulate a large range of flows. If we define a potential hazard area as any location inundated by at least one simulated flow in this MC approach, the hazard area would be 72~km$^2$. As the mapped flow area from the Tolbachik eruption is 39\% of this area, it would be more practical to use this as the \textit{a priori} hazard area because it more reasonably reflects the potential inundation area. Both the Kilburn-Kauahikaua method and this MC method are illustrated in Figure \ref{fig_hazardarea}.				
	
	\paragraph{Review of Validation Level 3} Instead of using model sensitivity and the Jaccard index as fitness values for the various lava flow models, now the two predictive values will be used. The potential hazard area is defined as the distribution of MC simulations (72~km$^2$). To give an example calculation, $\text{Pr}(Lava|Sim)$ is found with Equation \ref{eq_simplepost} by dividing true positives (green boxes in \ref{fig:tolbachik}) by the simulation area (green and red boxes). The NPV, $\text{Pr}(\neg Lava|\neg Sim)$, is found by diving true negatives (top right of Figure \ref{fig_2x2}), by the area not simulated (top half of Figure \ref{fig_2x2}).
	
		\begin{center}
		\textbf{Traditional Fit Metrics and Bayesian Posterior Functions}\\
		\begin{tabular}{l c c c c}
			\toprule
			Transition&\multicolumn{4}{c}{\textbf{Results of simulations over SRTM DEM}}\\
			Function& Jaccard & Sensitivity & Pos. Pred. Value & Neg. Pred. Value\\
			\midrule
			4/P/S & 56.7\%& 76.4\%& 70.4\%& 84.5\% \\
			8/P/S & 61.1  & 80.8  & 73.2  & 87.3   \\
			4/N/S & 57.2  & 77.5  & 70.3  & 85.1   \\
			8/N/S & 63.1  & 82.8  & 74.4  & 88.5   \\
			4/P/E & 51.2  & 71.5  & 66.0  & 81.4   \\
			8/P/E & 58.8  & 78.2  & 72.0  & 85.7   \\
			4/N/E & 54.5  & 74.5  & 68.7  & 83.3   \\
			8/N/E & 59.6  & 78.8  & 72.8  & 86.2   \\
			
			\bottomrule
		\end{tabular}
	\end{center}
	
	For the rest of this section, the 8/N/S (the 8-connected, no parent-child relationships, with slope-proportional spreading) algorithm will be used. This is preferred because it out-performed other algorithms over the SRTM DEM. By applying this model to the Tolbachik lava flows, Bayesian methods will be used to improve the ``Pulse Volume'' parameter and will later be used to constrain model uncertainty at Tolbachik.
	
	\subsection{Improving model performance on one model parameter}\label{sec_bayespulse}
	In the Tolbachik validation tests given as examples of comparing simulation algorithms against real lava flows (Section \ref {sec_tolb_bench}), all but one algorithm performed worse on the TanDEM-X derived elevation model. This was in part due to large run-out distances in the simulations (e.g. bottom right of Figure \ref{fig:tolbachik}), which considerably increased simulation false positives. The large run-out distances might be due to the pulse volume, the volume of lava given to source cells at each code loop in MOLASSES, being poorly chosen. Here, the Bayesian statistics defined above will be used to compare different pulse volumes and identify an optimal pulse volume. 
		
		An optimal pulse volume will ideally produce a flow simulation with the highest PPV and NPV. Pulse volumes with high associated PPVs will produce simulations where areas simulated as inundated by lava will have a high likelihood of actually being inundated by lava. Pulse volumes with high associated NPVs will produce simulations where areas simulated to not be inundated will hava a high likelihood of actually not being inundated.
		
		\subsubsection{Model Execution} 
		To populate $Sim$, I have run the MOLASSES lava flow code using TanDEM-X derived parameters listed in Table \ref{tab_parameters_pulsebayes}. All variables are fixed except the pulse volume parameter, which is the amount of lava delivered to source cells in the Cellular Automata grid of MOLASSES. The lowest pulse volume, 1755 m$^3$ per pulse, is approximately the product of the TanDEM-X DEM grid cell size (225~m$^2$) and the residual flow thickness (7.8~m). The other 15 pulse volumes are multiples of this volume (i.e. they are 1.5 to 8.5$\times$ 1775 m$^3$).
		
		\begin{table}[h!]
		\centering
			\caption{MOLASSES Flow Parameters}
			\begin{tabular}{l l}
				\toprule
				Elevation Model & 15-m bistatic TanDEM-X, 11 Nov 2015\\
				Modal Thickness & 7.8~m\\
				Pulse Volumes & 16 equally separated volumes, [1755,14917] m$^3$\\
				\midrule
				Vent$_N$ Easting & 582800~m (UTM Zone 57)\\
				Vent$_N$ Northing & 6182100~m\\
				Vent$_N$ Total Volume & 4.63$\cdot10^7$~m$^3$\\
				\midrule
				Vent$_S$ Easting & 582475~m\\
				Vent$_S$ Northing & 6180700~m\\
				Vent$_S$ Total Volume & 1.737$\cdot10^8$~m$^3$\\
				\bottomrule
			\end{tabular}
			\label{tab_parameters_pulsebayes}
		\end{table}
		

		Model output is compared to a list of x,y locations in the Tolbachik area that have been inundated or not. This location list is stored in a raster with the same projection and extent as the elevation model used in MOLASSES. ASCII locations output by MOLASSES are also listed in the same projection within the same extent as the elevation model. This enables direct comparison between the Model information (i.e. $Sim$) and the mapped lava flow (i.e. $Lava$). True Positives, False Positives, and False Negatives are reported as cell counts (number of grid locations where $Lava$ and $Sim$ agree or not). Three examples of these simulations are mapped in Figure \ref{fig:pulse_map}.

		\begin{figure}
		\centering
		\includegraphics[width=\linewidth]{\FigPath/pulse_examples_72dpi}
		\caption[MOLASSES Simulations with different Pulse Volume parameter values of the 2012-3 Tolbachik Lava Flows]{MOLASSES Simulations of the 2012-3 Tolbachik Lava Flows. Vents are shown as blue triangles and the mapped flow is outlined in black. a) Pulse Volume = 1755 m$^3$, the simulation far exceeds the true runout distance; b) Pulse Volume = 4387 m$^3$, this simulation performs best under the NPV $\text{Pr}(\neg Lava|\neg Sim)$ test; c) Pulse Volume = 14040 m$^3$, this simulation performs best under the PPV $\text{Pr}(Lava|Sim)$ test, but does not have a runout length similar to the mapped flow.}
		\label{fig:pulse_map}
		\end{figure}

		\subsubsection{Results}
		
		Three example simulations are shown in Figure \ref{fig:pulse_map} using simulation parameters from Table \ref{tab_parameters_pulsebayes} and different pulse volume values. With increased pulse volume, simulated run-out distance is shorter. This is because the MOLASSES code ends once all volume is delivered to the vents and the DISTRIBUTE module has run once more. In other words, if the pulse volume is doubled, the number of times the PULSE and DISTRIBUTE modules will be run will be halved, as the total volume will be delivered to the vents in half the code loops (see Figure \ref{fig_flowchart}). By running DISTRIBUTE fewer times, cells have fewer opportunities to advect lava downslope.

		The positive predictive value is the fundamental tool of Bayesian statistics, and quantifying it enables an update of belief in risk of lava inundation. A perfect PPV would mean that if the model simulates lava inundating a location, lava will certainly inundate that location. PPV is calculated for simulated lava flows of different Pulse Volumes and is graphed in Figure \ref{fig_lavaGsim}. From this, it can be seen that the highest pulse volumes, which coincidentally form the shortest flow simulations, perform best with this test, with the best fit having a pulse volume of 14040~m$^3$ per algorithm loop (Figure \ref{fig:pulse_map},c). A local maximum does exist in the low pulse volumes at 4387~m$^3$ per loop.


		\begin{figure}[h!]
			\centering
			\begin{gnuplot}[terminal=latex, terminaloptions=rotate]
				unset key
				set size 0.7,0.7
				set format xy "$%g$"
				set xlabel "Pulse Volume (m$^3$)" rotate by 90
				set ylabel "Pr$(Lava|Sim)$"
				set ytics 0.05
				set xtics 4000
				plot "data/results_bayes.dat" using 1:2 with linespoints lt 4 pt 7
			\end{gnuplot}
			\caption{PPV, $\text{Pr}(Lava|Sim)$, for MOLASSES flows with differing Pulse Volumes.}
			\label{fig_lavaGsim}
		\end{figure}
		
		The negative predictive value $\text{Pr}(\neg Lava|\neg Sim)$, is the percentage of non-inundated area in the simulation that is also not inundated in real life. A perfect NPV would indicate that, if a model does not simulate a hit for a location, lava will certainly not inundate that location. The NPV of simulations with different pulse values are shown in Figure \ref{fig_neglavaGsim}. Unlike the previous predictive value analyzed, the best performing flows, according to NPV, have smaller pulse volumes and the best performing volume is 4387~m$^3$ per model pulse loop (Figure \ref{fig:pulse_map},b).

		\begin{figure}[h!]
			\centering
			\begin{gnuplot}[terminal=latex, terminaloptions=rotate]
				unset key
				set size 0.7,0.7
				set format xy "$%g$"
				set xlabel "Pulse Volume (m$^3$)" rotate by 90
				set ylabel "Pr(not $Lava|$not $Sim)$"
				set ytics 0.01
				set xtics 4000
				plot "data/results_bayes.dat" using 1:3 with linespoints lt 4 pt 7
			\end{gnuplot}
			\caption{NPV, $\text{Pr}(\neg Lava|\neg Sim)$, for MOLASSES flows with differing Pulse Volumes.}
			\label{fig_neglavaGsim}
		\end{figure}
	
	\subsection{Incorporating Model Uncertainty with a Monte Carlo method}\label{sec_MC}
		Model uncertainty is a result of input parameter uncertainty, such as uncertainty in the underlying DEM. This can be distinguished from model error, which might be defined as the difference between the true lava flow and a simulation carried out with perfect input parameters, and is created by the inherent deviations between a computer model and real life processes. Because there is parameter uncertainty, it is essential to quantify the range of model solutions given the likely range of each parameter.
		
		In this example, elevation uncertainty will be examined. Elevation uncertainty is an element in the MOLASSES module \textbf{INITFLOW}, where each grid cell elevation can be defined randomly before the lava flow simulation begins. The user can add an elevation uncertainty, in meters, to the configuration file. If this value is provided, each grid cell will receive a new elevation value randomly selected from a normal distribution whose mean is the DEM elevation and the standard deviation is the uncertainty value.
		
		\begin{table}[h!]
		\centering
			\caption{Monte Carlo MOLASSES Flow Parameters}
			\begin{tabular}{l l}
				\toprule
				Elevation Model & 75-m SRTM\\
				Elevation Uncertainty, $1\sigma$ & 3~m\\
				Residual Thickness & 7.8~m\\
				Pulse Volume & 44200 m$^3$\\
				\midrule
				Vent$_N$ Easting & 582800~m (UTM Zone 57)\\
				Vent$_N$ Northing & 6182100~m\\
				Vent$_N$ Total Volume & 4.63$\cdot10^7$~m$^3$\\
				\midrule
				Vent$_S$ Easting & 582475~m\\
				Vent$_S$ Northing & 6180700~m\\
				Vent$_S$ Total Volume & 1.737$\cdot10^8$~m$^3$\\
				\bottomrule
			\end{tabular}
			\label{tab_parameters_MC}
		\end{table}
		
		%Map of the MC flows
		\begin{figure}
			\centering
			\includegraphics[width=0.7\linewidth]{\FigPath/MC_probmap_300dpi}
			\caption[Map of Monte Carlo simulations of the 2012-3 Tolbachik lava flows]{Cumulative distribution of 100,000 simulated lava flows over SRTM topography with 3~m elevation uncertainty. The red outline is the mapped flow extent of the 2012-3 Tolbachik flow. Darker purple areas represent more simulation hits (i.e. higher $\text{Pr}(Sim)$).}
			\label{fig_MC_map}
		\end{figure}
		
		The Monte Carlo method runs MOLASSES 100,000 times over a 75-m SRTM DEM. Vertical uncertainty of this data is estimated by \citet{rodriguez2006global} for Eurasia to be 6.2~m at a 90\% confidence level and is shown to be randomly distributed. With this result, elevation uncertainty in the MOLASSES model is given a value of $1\sigma=3$~m. Other input parameters remain unchanged from the validation exercise above; MOLASSES flow parameters for the Monte Carlo model are listed in Table \ref{tab_parameters_MC}. The combined 100,000 simulations are mapped in Figure \ref{fig_MC_map} where flow color indicates the number of flows that impacted each location.
		

				
		\subsubsection{Bayesian distribution of MC results}
			The reliability of a model can be better understood by showing the distribution of model performance given model uncertainty, as opposed to treating model parameters, and thus model output, as completely certain.  Figure \ref{fig:MC_dist} shows the distribution of PPV and NPV scores. Each dot in the main chart of Figure \ref{fig:MC_dist} represents a single flow simulation over a partially randomly generated DEM. The clustering of these points shows a positive correlation between the two predictive values, and both metrics are not normally distributed as shown on the histograms on either side of the main chart of Figure \ref{fig:MC_dist}. The flow simulation assuming no elevation uncertainty (shown in the top right corner of Figure \ref{fig:tolbachik}) fits the mapped flow better than the median value of both predictive values in the distribution, though it still lays within the MC distribution.
			
		%Graph of the performances
		\begin{figure}[h!]
			\centering
			\includegraphics[width=0.7\linewidth]{\FigPath/MC_posteriors_300dpi}
			\caption[Fitness statistic distribution of Monte Carlo simulations of the 2012-3 Tolbachik lava flows]{Fitness statistic distribution for 100,000 simulations of the 2012-3 Tolbachik Lava Flows, over SRTM topography with 3~m standard elevation uncertainty. Each point represents the predictive values of inundation/non-inundation for one simulation. Red lines are the fitness values of a simulated flow over SRTM data assuming 0~m elevation uncertainty (top right corner of Figure \ref{fig:tolbachik}). Black lines are placed at the median values of each predictive value.}
			\label{fig:MC_dist}
		\end{figure}
			
			If the elevation model used were perfect, the predictive values would be a single number ($\text{Pr}(Lava|Sim)=74.4$\% and $\text{Pr}(\neg Lava|\neg Sim)=88.5$\%). However, because elevation values have inherent uncertainty, model fitness, as defined by the predictive values, can be given as a range. Including elevation uncertainty, the $\text{Pr}(Lava|Sim)$ fitness has a range of 59-76\% and the $\text{Pr}(\neg Lava|\neg Sim)$ has a range of 77-91\%.


		\subsubsection{Estimating inundation risk from the simulated frequency of inundation}
		Figure \ref{fig_MC_map} shows a map view of the probability of inundation from the 100,000 MC simulations. Generally, areas within the mapped flow appear to be inundated by more simulations than outside the mapped flow. But can the probability of simulation inundation, $\text{Pr}(Sim)$, be used in a more formal way to judge the probability of lava inundation? In this section, the Tolbachik region map will be split into sub-regions based on $\text{Pr}(Sim)$, and the probabilities of inundation and not inundation will be compared.
		
		Three example sub-regions are shown in Figure \ref{fig_MCsubregions}. These sub-regions are defined as the area where $\text{Pr}(Sim)$ falls between a 10\% range. For instance, the top sub-region in Figure \ref{fig_MCsubregions} shows all locations that were forecast as inundated by at least 90\% of all MC simulations, while the middle sub-region example contains all locations inundated by 40-50\% of simulations. The sub-regions indundated by 0-10\% and 90-100\% of simulations are the largest sub-regions, while other sub-regions are thin rings between these two bounding sub-regions.
		
		\begin{figure}[!h]
			\centering
			\includegraphics[width=0.7\linewidth]{\FigPath/PR_sim_examples_300dpi}
			\caption[Sub-regions within the Monte Carlo solution, defined by $\text{Pr}(Sim)$]{Sub-regions within the Monte Carlo solution, defined by $\text{Pr}(Sim)$. At the top, all areas inundated by $\ge$90\% of all simulations are shown in green (True Positives) and orange (False Positives). The remaining mapped lava flow (False Negatives) is red. The underlying region is the total Monte Carlo Hazard Area. At center, A thin band represents all areas inundated by 40-50\% of all Simulations. At bottom, a thick shell of areas rarely inundated by simulated lava is mostly orange, indicating rarely simulated flow areas are unlikely to have been inundated in real life.}
			\label{fig_MCsubregions}
		\end{figure}	
		
		It can be seen in Figure \ref{fig_MCsubregions} that most of the mapped flow is covered by the $\text{Pr}(Sim\ge90\%)$ sub-region, as the green true positive region is larger than the red false negative region. The sub-region defined as $\text{Pr}(0<Sim<10\%)$ does not spatially intersect with the mapped flow area as much as the $\text{Pr}(Sim\ge90\%)$ sub-region, and this can be seen in the lower right of Figure \ref{fig_MCsubregions} as the sub-region area is mostly orange false positives with small green true positive areas.
		
		The relative risk of inundation can be calculated for each subregion by comparing the probability of actual flow inundation $\text{Pr}(Lava)$ against the probability of not inundation $\text{Pr}(\neg Lava)$ using the Bayes Factor of Equation \ref{eq_BF}. Here the Evidence provided by the MC simulation is the probability of simulated inundation $\text{Pr}(Sim=\text{X})$ for a given area. Hypothesis 1 is that an area with a given probability of simulated inundation can be described as inundated by actual lava (``Inundation''). The opposing and complementary Hypothesis 2 is that this area can be described as not being inundated by actual lava (``Not Inundation''). For example, the relative probability of inundation for the sub-region defined by $\text{Pr}(40\le Sim<50\%)$ can be given as
		\begin{equation}
			\text{BF} = \frac{\text{Pr}(0.4\le Sim<0.5~|~\text{Inundation})}{\text{Pr}(0.4\le Sim<0.5~|~\text{Not~Inundation})}.
		\end{equation}
		The numerator $\text{Pr}(0.4\le Sim<0.5~|~\text{Inundation})$ is the probability of 40-50\% of simulations hitting a given location, given that location is actually engulfed by lava. Graphically, this is the percent of the true flow (red and green areas in the center example of Figure \ref{fig_MCsubregions}) that are within the simulated sub-region (True Positive green areas in the center example of Figure \ref{fig_MCsubregions}). The denominator $\text{Pr}(0.4\le Sim<0.5~|~\text{Not Inundation})$ is probability of 40-50\% of simulations hitting a location, given that location is not actually inundated by the real flow. This is the percent of the area not hit by the flow (light gray and orange areas in Figure \ref{fig_MCsubregions}) that is within the False Positives subregion (orange areas in Figure \ref{fig_MCsubregions}). 
		
		The probability that 40-50\% of simulations hit a location that is inundated is 2.7\%. The probability the 40-50\% of simulations hit a location that is not inundated by lava is 2.1\%. The Bayes Factor is then calculated to be 1.3, where the model of Inundation is 1.3 times more likely to describe this subregion than the model of Not Inundation. Referring to Table \ref{tab_BFinterps}, this result means that the preference for Inundation over Not Inundation is ``just worth a mention.'' Results for each 10\% wide sub-region are given in Table \ref{tab_BFresults} and are illustrated in Figure \ref{fig_bayesfactor}.
		

		\begin{table}
			\centering
			\caption{Relative likelihood of inundation given $\text{Pr}(Sim)$}
			\begin{tabular}{l c c c l}
				\toprule
				 & $\text{Pr}(Sim$=X$|$ & $\text{Pr}(Sim$=X$|$ & Bayes & \citet{jeffreys1998theory}\\
				X & $Lava)$ & $\neg Lava)$ & Factor & Interpretation \\
				\midrule
				$0<$X$<0.1$ & 0.06 & 0.67 & 0.09 & Strong evidence against inundation\\
				$0.1\le$X$<0.2$ & 0.02 & 0.08 & 0.23 & Substantial evidence against inund.\\
				$0.2\le$X$<0.3$ & 0.03 & 0.04 & 0.74 & No Inundation more likely\\
				$0.3\le$X$<0.4$ & 0.03 & 0.03 & 1.21 & Inundation more likely than not\\
				$0.4\le$X$<0.5$ & 0.03 & 0.02 & 1.29 & Inundation more likely than not\\
				$0.5\le$X$<0.6$ & 0.03 & 0.02 & 1.53 & Inundation more likely than not\\
				$0.6\le$X$<0.7$ & 0.04 & 0.01 & 2.63 & Inundation more likely than not\\
				$0.7\le$X$<0.8$ & 0.05 & 0.02 & 2.69 & Inundation more likely than not\\
				$0.8\le$X$<0.9$ & 0.06 & 0.02 & 2.84 & Inundation more likely than not\\
				$0.9\le$X$\le 1.0$ & 0.66 & 0.10 & 6.93 & Substantial evidence for inundation\\
				\bottomrule
			\end{tabular}
			\label{tab_BFresults}
		\end{table}
		
		\begin{figure}[!h]
			\centering
			\includegraphics[width=0.7\linewidth]{\FigPath/LR_bayes}
			\caption[Relative likelihood of inundation given $\text{Pr}(Sim)$]{Relative likelihood of inundation given $\text{Pr}(Sim)$. Only locations inundated by $<10\%$ of Monte Carlo simulations show strong evidence against inundation in the actual 2012-3 Tolbachik lava flows.}
			\label{fig_bayesfactor}
		\end{figure}
		
		Only the sub-region least likely to be hit by simulations, where $\text{Pr}(Sim<10\%)$, has a Bayes Factor of $<10^{-1}$, which is interpreted as strong evidence against lava flow inundation. As the Bayes Factor can be treated as posterior odds for or against a model \citep{aspinall2003evidence}, a factor of $1/10$ indicates 10:1 odds against inundation. Sub-regions with factors greater than $1/10$, and are therefore more in support of flow inundation, have odds less than 10:1 against inundation. As 10:1 odds against inundation is the same as a probability of 9\% for inundation, all sub-regions besides the $\text{Pr}(Sim<10\%)$ sub-region have $\text{Pr}(Lava>9\%)$.
		


%%%%%%%%%%%%%%%%%%%%%%%%%%%%%%%%%%%%%%%%%%%%%%%%
%DISCUSSION
%%%%%%%%%%%%%%%%%%%%%%%%%%%%%%%%%%%%%%%%%%%%%%%%
%%%%%%%%%%%%%%%%%%%%%%%%%%%%%%%%%%%%%%%%%%%%%%%%
%%%%%%%%%%%%%%%%%%%%%%%%%%%%%%%%%%%%%%%%%%%%%%%%

\section{Discussion}\label{sec:discussion}
	\subsection{Validation of MOLASSES models}
	Unique flow algorithms are validated by using common tests that show whether these algorithms replicate expected lava flow morphologies under specific conditions. Different tests that are applicable to CA codes fall into three validation levels: 1) tests that show that simulations don't change when parameters remain meaningfully identical; 2) tests that show that simulations replicate experimental results or analytical expectations; and 3) tests that show that simulations replicate real world flow examples. A summary of the results of the eight algorithms against the example validation exercises are given below in Table \ref{tab_algorithmresults}.
		
	\begin{table}[h]
		\centering
		\caption{Transition Algorithm Results}
		\begin{tabular}{l | c | c | c c}
			\toprule
			Transition&\multicolumn{4}{c}{\textbf{Levels}}\\
			Function&1&2&\multicolumn{2}{c}{3}\\
			& DEM Rotation & Pancake & SRTM & TanDEM-X \\
			\midrule
			\textbf{4/P/S} & Pass & Fail & Pass & Pass\\
			\textbf{8/P/S} & Pass & Pass & Pass & Fail\\
			\textbf{4/N/S} & Fail & Fail & Pass & Fail\\
			\textbf{8/N/S} & Pass & Pass& Pass & Fail\\
			\textbf{4/P/E} & Fail & Ambiguous & Pass & Pass\\
			\textbf{8/P/E} & Fail & Ambiguous & Pass & Pass\\
			\textbf{4/N/E} & Fail & Pass & Pass & Pass\\
			\textbf{8/N/E} & Fail & Pass & Pass & Pass\\
			
			\bottomrule
		\end{tabular}
		\label{tab_algorithmresults}
	\end{table}
	
	Because these validation levels increase in complexity from Level 1 to Level 3, one possible strategy in validating different algorithms would be to only test algorithms at more complex levels after they successfully pass less complex tests. Valid models might then be determined by elimination. Only 3 of 8 tested algorithms pass the first ``rotating slope'' exercise: 4/P/S, 8/P/S, and 8/N/S. Although more algorithms passed the second ``bingham flow on a flat surface'' exercise, only 8/P/S and 8/N/S passed the previous exercise and this one. Both of these flows then successfully replicated the Tolbachik lava flows over SRTM topography. Therefore the 8/P/S and 8/N/S algorithms hold up to three tests.
	
	Overall, in all tests 8-connected models outperform 4-connected models. While equal sharing algorithms outperform slope-proportional sharing on a flat slope, they fail on a rotating DEM and perform about the same on real topography. There does not seem to be an unambiguously better choice between using parent-child relationships or not. If future tests continue to show similar performance between models with and without parentage, other reasons can be used to choose a model, such as computer run-time. 
	
	The strength of the MOLASSES code is that new algorithms, such as those used in the SCIARA model \citep{crisci2004simulation}, can be implemented relatively quickly and run through the validation exercises, which are written in Python. Combinations of implementation strategies can also be created on the fly by adjusting the makefile of the MOLASSES code instead of the code itself.

	\subsection{Bayesian applications for real lava flows}
		Validation of lava flow models is important as a method of increasing the value of models to forecast lava flow processes, thereby decreasing preventable loss. Flow models are generally improved by reducing false positives and false negatives while increasing the true positive area, which is the union of a flow simulation and real, mapped lava flows. Often, reducing one type of error comes at the cost of increasing another type. For example, by increasing the pulse volume parameter, false positives can be reduced while false negatives are increased (Figure \ref{fig:pulse_map}).
		
		A decision can be made as to whether false positives or false negatives are more important to reduce in calibrating a lava flow model. One possible decision would be to prefer the reduction of false negatives over the reduction of false positives, with the reason being that a forecast of inundation without eventual engulfment by lava (a false positive result) is bad, but an unexpected engulfment by lava (a false negative) would be worse.
		
		\subsubsection{Using Bayesian statistics to compare models}
		In Section \ref{sec_bayespulse}, an optimal pulse volume was defined as a volume that produced a simulated lava flow that had the highest PPV and NPV. However, the pulse volume 14040 m$^3$ had the highest PPV, while the pulse volume 4387 m$^3$ had the highest NPV. The simple definition of ``optimal'' pulse volume does not seem to work.
		
		The best pulse volume might be decided based on a preference for reducing false negatives as discussed above. The PPV of inundation, $\text{Pr}(Lava|Sim)$ increases with increased true positives and decreases with increased false positives (Equation \ref{eq_simplepost}). It is essentially blind to false negatives. The NPV $\text{Pr}(\neg Lava|\neg Sim)$ is the opposite, increasing with increased true negative results, while decreasing with increased false negative results (Equation \ref{eq_simplenegpost}). 
		
		If false negatives are more important to reduce than false positives, more weight can be given to the NPV of a simulation than the PPV. In Figure \ref{fig_weightedposteriors}, the PPV and NPV for each pulse volume (Figures \ref{fig_lavaGsim} and \ref{fig_neglavaGsim}) are used to produce weighted average scores of pulse volume. The extent to which false negative reduction might be preferred over false positive reduction is unknown, so four hypothetical weights are used. One of the weighted averages that is plotted assumes both predictive values are equally important in determining the best pulse volume parameter. The other three curves weigh NPV as being 2, 5, and 10 times as important as PPV in scoring pulse volumes. Using these three weights, the highest scoring pulse volume is 4387 m$^3$ (Figure \ref{fig:pulse_map}b). If the two values have equal importance, the highest scoring pulse volume is 14040 m$^3$ (Figure \ref{fig:pulse_map}c).
		
		\begin{figure}[h!]
			\centering
			\begin{gnuplot}[terminal=latex, terminaloptions=rotate]
				unset key
				set size 0.7,0.7
				set format xy "$%g$"
				set xlabel "Pulse Volume (m$^3$)" rotate by 90
				set ylabel "Model Score"
				set ytics 0.02
				set xtics 4000
				plot "data/results_bayes.dat" using 1:7 with points pt 1, "data/results_bayes.dat" using 1:5 with points pt 7, "data/results_bayes.dat" using 1:6 with points pt 3, "data/results_bayes.dat" using 1:4 with points pt 6
			\end{gnuplot}
			\caption[Weighted averages of $\text{Pr}(Lava|Sim)$ and $\text{Pr}(\neg Lava|\neg Sim)$ for different pulse volumes]{Weighted averages of $\text{Pr}(Lava|Sim)$ and $\text{Pr}(\neg Lava|\neg Sim)$ for different pulse volumes. Hollow circles, NPV ($\text{Pr}(Lava|Sim)$) is given 10$\times$ the weight of the PPV ($\text{Pr}(\neg Lava|\neg Sim)$); solid circles, 5$\times$ weight; asterisks, 2$\times$ weight; plusses, equal weight between predictive values. The highest scoring pulse volume is 4387 m$^3$ for all weight ratios except when both values are weighted equally.}
			\label{fig_weightedposteriors}
		\end{figure}
		
	\subsubsection{Decision Making with the Bayes Factor}
	By dividing the Monte Carlo set of simulations of the Tolbachik lava flows into sub-regions based on $\text{Pr}(Sim)$, the relative likelihood of inundation was calculated (Figure \ref{fig_bayesfactor}). By using the Bayes Factor to compare likelihood of Indundation to Not Inundation, most sub-regions were not significantly better described by one model or the other. Only the sub-region of locations least likely to be hit by flow simulations had ``strong evidence'' against inundation, and only the sub-region of locations most likely to be hit by simulations had ``substantial risk'' of inundation. These results leave a large amount of ambiguity when making decisions to fortify or evacuate due to a lava flow.
	
	In discussing the decision making process for calling an evacuation due to an eruption at Mount Vesuvius, \citet{marzocchi2007probabilistic} provided a cost-loss model where two actions can be taken, protect and do not protect. Two costs are associated with these actions: $\mathcal{C}$ is the cost of protection and $\mathcal{L}$ is the cost of loss if the volcanic hazard occurs while a decision to not protect is made. If $\mathcal{L}$ is incurred, it is assumed to exceed cost $\mathcal{C}$, often because loss due to volcanic hazards includes the loss of life.
	
	\citet{marzocchi2007probabilistic} show that the minimum cost can be achieved if protection occurs when $p>\mathcal{C}/\mathcal{L}$ where $p$ is the probability of the hazard occuring. The ratio $\mathcal{C}/\mathcal{L}$ is hard to quantify because the socio-economic cost of lost lives or of lost trust in the government (in the case of false evacuations) is difficult to calculate, even though the physical cost of evacuation and the cost of infrastructure loss might be straight forward estimates. \citet{woo2008probabilistic} provided one estimate of $\mathcal{C}/\mathcal{L}=0.1$ at the very maximum if 10\% of people evacuated from an area would owe their lives to that evacuation. In this case, protection would be made when $p>0.1$.
	
	While lives are not commonly lost due to lava flows with the notable exceptions of Laki (A.D. 1783, Iceland) and Nyiragongo (A.D. 1977 and 2002, Democratic Republic of the Congo) \citep{peterson2000lava}, the ``protection threshold'' $p>0.1$ can still be used as an example. All Monte Carlo sub-regions have a probability of inundation ($\text{Pr}(Lava)$) greater than 0.1 except the sub-region $\text{Pr}(Sim<0.1)$ (Figure \ref{fig_bayesfactor}). If action to protect is made at $\text{Pr}(Lava>0.1)$, then evacuations or fortifications will be made for all locations inundated by $\ge$10\% of Monte Carlo simulations.
	
	Two Bayes-factor thresholds in Figure \ref{fig_bayesfactor} have odds against lava inundation at better than 10:1 (i.e. $\text{Pr}(Lava<0.1)$), ``strong evidence'' and ``very strong evidence'' against inundation. Given the hypothetical protection threshold of $p>0.1$, the decision to protect against a lava flow should be made for a location whenever the Bayes factor supporting inundation is $>10^{-1}$, or when there is less than ``strong evidence'' against inundation for that area. If the protection threshold is decided to be less than 0.1, then the evidence against inundation will have to be even stronger to support a decision to not protect an area. 
	
	It is unknown whether sub-regions defined as $\text{Pr}(Sim<0.1)$ should be expected to be relatively safe areas when using Monte Carlo results at future lava flow sites. To test the effectiveness of the MOLASSES algorithm used for the Tolbachik flows in other study areas, more example flows around the world will have to be tested.
		
		
\section{Conclusions}
	The MOLASSES framework used in this project provides a modular set of functions that can be interchanged to fundamentally alter the flow simulation algorithm. While only 8 different algorithms, which focused on simple variations in the transition function of Cellular Automata codes, were explored, different modules can feasibly be modified to change other flow characteristics (e.g. vent geometry, mass flux at source locations, temperature dependent flow).
	
	All CA and CA-like codes, including SCIARA \citep{crisci2004simulation}, MAGFLOW \citep{del2008simulations}, ELFM \citep{damiani2006lava}, LavaPL \citep{connor2012probabilistic}, and MOLASSES share similarities. For instance, lava inundation is a binary condition for each location on a grid and source locations are set at defined grid cell coordinates. This enables a common set of validation exercises to be defined, which can be used to test the validity of all CA codes.
	
	Three validation levels have been identified and different tests have been given which apply to each level. After a code is successfully demonstrated to conserve mass, the first validation level ensures that the code does not change when parameters are changed in a meaningless way. The example exercise is that simulated flows should not be slope-direction dependent. The second level compares simulations to analytical or experimental results. As lava flows on large scales are similar to Bingham fluids, a flow simulator should produce a circular pattern on a flat surface. The third and final level tests simulations against true lava flows. Here, the 2012-3 Tolbachik lava flow is used as an example, with flow parameters identified using TanDEM-X analysis \citep{kubanek2015lava}. It might only make sense to validate codes at higher validation levels once they have been sufficiently developed to be validated at lower levels.
	
	Two common fitness tests for lava flows are the Jaccard coefficient and model sensitivity. Model sensitivity gives the percentage of a mapped lava flow that a simulation successfully forecasts. A possible alternative to these tests is to use Bayesian posterior probabilities, or predictive values to evaluate the positive and negative predictive values of the simulation. These predictive values (PPV, $\text{Pr}(Lava|Sim)$; and NPV, $\text{Pr}(\neg Lava|\neg Sim)$) give the likelihood of flow inundation or not inundation given the simulation results.
	
	Once lava simulators have been successfully validated, these Bayesian statistics can be used to improve model input parameters and evaluate model uncertainty by incorporating uncertainty inherent in input parameters. It is also possible that the two predictive values can be used to aid in decision making processes associated with active lava flow crises.
	
	
\section{Data Statement}
This code is available for free use on GitHub at the USFVolcanology page located at \url{https://github.com/USFvolcanology}, while the validation codes can be found at \url{https://github.com/jarichardson/MOLASSES_benchmarking}. The MOLASSES code and the validation algorithms are kept in seperate self-contained repositories.

\section{Notation}
	\begin{table}[h!]
		\centering
		\caption{Notation}
		\begin{tabular}{l l}
			\toprule
			\textbf{A}& Cellular Automata (CA) structure\\
			E$^2$& set of CA point locations\\
			V & set of source locations in E$^2$\\
			X & local neighborhood for each cell\\
			$n$ & an element of X\\
			S & set of subsets for each cell\\
			$\sigma$ & transition function of the CA\\
			$\gamma$ & source function of the CA\\
			$i$,$j$ & row, column location within E$^2$\\
			$c$ & the location of a central cell of interest\\
			$t$ & timestep in the CA\\
			S$_e$, S$_h$, S$_h0$ & elevation, lava thickness, and critical thickness of a cell\\
			V$_{in}$ & total volume delivered to source locations\\
			V$_{out}$ & total volume in all inundated cells\\
			$A_{flow}$ & total footprint area of a flow simulation\\
			$d_{max}$ & maximum distance of lava from the source location\\
			$N$ & the predefined hazard area\\
			$R_{max}$ & maximum hazard radius\\
			$\rho$ & density of lava flow crust\\
			$\epsilon$ & emprical extension-before-failure value\\
			$S$ & tensile crust strength\\
			$Q$ & mean volumetric flux\\
			$g$ & acceleration due to gravity\\
			$\kappa$ & bulk thermal diffusivity\\
			$\mathcal{C}$ & the cost of protection\\
			$\mathcal{L}$ & the cost of loss if protection does not occur\\
			$p$ & probability of hazard occuring\\
			\bottomrule
		\end{tabular}
		\label{tab_notation}
	\end{table}

%\section{Acknowledgments}
%The development of this code was supported by SSI Grant

%%%%%%%%%%%%%%%%%%%%%
%%References Section
%\section{References}
\nobibliography{dissertation_refs}
\bibliographystyle{apalike} 

%\begin{figure}
%\centering
%\includegraphics[width=\linewidth]{map_diff}
%\label{fig:map_diff}
%\end{figure}

