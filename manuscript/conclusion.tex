\chapter[Conclusion]{Conclusion}

%Chapter 1, Sills
%Chapter 2, Molasses
%Chapter 3, Syria Planum
%Chapter 4, Arsia Mons 
%Chapter 5, Spatial Density
Five facets of distributed volcanism have been explored in this dissertation: the igneous intrusion network of a volcanic field, lava flows, the spatial and temporal evolution of a shield field, the recurrence rate and magma delivery rate of volcanism, and the spatial intensity of vents in volcano clusters. 

In Chapter \ref{ch_sills}, seven sills are identified with the aid of three lidar surveys in the eroded San Rafael Volcanic Field, Utah (USA). Their volumes range from 10$^{-4}$-10$^{-1}$~km$^3$ while their dimensions range from centimeters to 40~m in thickness and 1-10~km in planform width. This tabular thickness to diameter ratio implies that the sills each formed from one period of injection from one dike, as no sills thicken into a lacolith type body \citep{gudmundsson2012magma}. The total volume of sills represents $>$92\% of the total magma stored at this depth, with the rest stored in dikes and volcanic conduits. Comparing the volume of seven sills with a hypothetical volume of 0.1~km$^3$ of magma erupted per conduit, the sill volume in the 500~m high study area is one-quarter the volume delivered to the surface in the same area. At least one sill is identified to have been emplaced during or immediately preceding a volcanic eruption. The once co-magmatic connection of the sill with an adjacent conduit would have caused bubbles to concentrate either in the sill's magma or in the conduit's, depending on the volume flux of the ascending magma. This would have had a modulating effect on the eruption style of the volcano.

In Chapter \ref{ch_molasses}, I presented modular Cellular Automata software that simulates lava flows based on the algorithm from \citet{connor2012probabilistic}, along with example implementations created by modifying small portions of the simulated flow behavior. Some of these implementations are validated under specific conditions that are based on analytical solutions or real lava flows. We show that using the Bayesian posteriors known as the Positive Predictive Value and the Negative Predictive Value of a model, the probability that the lava flow simulator will meaningfully forecast danger or safety is more directly quantified. Also, by adding known uncertainty in input parameters, such as the topographic error of a digital elevation model, the model output uncertainty is also able to be quantified. Decisions on whether to protect against lava can be made by identifying locations where inundation or safety is more certain, compared to locations whose probability of inundation widely varies when input parameters are adjusted within their uncertainty thresholds.

In Chapter \ref{ch_syria}, 263 volcanic vents have been cataloged in the Syria Planum region of Mars, with low shield edifices 100s~m in height and several to 10s~km in diameter. The volcanic unit that is lowest stratigraphically, while still visible at the surface is a group of lava flows emminating from a central vent volcano called Syria Mons. Volcanism then changed styles and formed a patchwork of low shield volcanoes to the east. North of this patchwork is the youngest cluster of volcanoes, and lavas from this northern cluster filled in grabens that cut the entire southern cluster. Three vent alignments in the catalog have been identified using the 2-point azimuth method of \citet{Cebria2011}, which point to previously identified tectonic centers in the Tharsis region \citep{Anderson2004}. The presence of these alignments might mean that magma exploited fractures in the crust related to the tectonic centers. Crater retention rate age-dating shows that the field was active from 3.5-2.6~Ga, spanning the Hesperian Period and continuing into the Early Amazonian Period. The modeled ages from crater age-dating do not seperate each of the three identified events, but does show a progression through time that agrees with the stratigraphic interpretation.

Chapter \ref{ch_arsia} identified a waning of volcanism in the Arsia Mons Caldera in the Tharsis Volcanic Province, Mars. The most recent volcanism in the caldera is a volcano cluster with 29 volcanic vents, which each have lava flows extending downhill for several~km. Each volcanic vent has been independently dated by counting craters on the lavas to have ages between 70-320~Ma, but these dates are highly uncertain. In addition to modeled age-dates, the lava flows abut each other providing stratigraphic information. These two age estimation methods have been consolidated to model the recurrence rate of volcanism in the caldera. At 150~Ma, the recurrence rate is modeled to have been 0.1-1 event per million years, with a median value of 0.25 events per million years. Since then, recurrence rate has monotonically decreased. This might be the tail end of larger volcanic activity, since 200~Ma-aged ashes are seen on the flanks of Arsia \citep{kadish2014middle}. By modeling volumes of the mapped lavas, the delivery rate of magma is also expected to have plateaued at 150~Ma at 0.1-3 km$^3$~Myr$^{-3}$ and remained at this level until 100~Ma, when the flux is modeled to decrease dramatically.

In Chapter \ref{ch_kde}, Kernel Density Estimation is used to model the spatial distribution of volcanic vent density in several volcano clusters on Earth, Venus, and Mars. By calculating the average vent intensity, the number of volcanic vents per km$^2$, for each volcano cluster, it is seen that distributed volcanism occurs on separate scales on each planet. Clusters on Earth are the most dense among the three planets, where vents are packed at 0.1 vents~km$^{-2}$. Clusters are least dense on Mars, where vents are two orders of magnitude more dispersed, at 0.001 vents~km$^{-2}$. Clusters on Venus have an intermediate density of 0.01 vents~km$^{-2}$. Clusters on all planets have the same range of vent prevalence, with tens to a few hundred vents being cataloged on all clusters. The exception to this is the Venus shield plains, which have $\sim$10,000 vents within their boundaries. The kernel bandwidth, an elliptical Gaussian distribution modeled for each volcano cluster to calculate spatial density, is indicative of anisotropy of vent production in the cluster. The orientation and elongation of the bandwidth can be related to geologic processes that control magma focusing in the lithosphere, including source region geometry, source evolution over time, pre-existing fractures, and geodynamic stress.

%%%%%%%%%%%%%%%%%%%%%%%%%%%%%%%%
%References
%%%%%%%%%%%%%%%%%%%%%%%%%%%%%%%%

%%%%%%%%%%%%%%%%%%%%%
%%References Section
%\section{References}

%\nobibliography{dissertation_refs}
%\bibliographystyle{apalike} 
